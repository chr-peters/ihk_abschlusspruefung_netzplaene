\documentclass[ngerman, t]{beamer}

\usepackage[utf8]{inputenc}
\usepackage[T1]{fontenc}
\usepackage[ngerman]{babel}
\usepackage[ngerman]{struktex}
\usepackage{pgf-umlcd}

\usetheme{Juelich}

\title{Automatisierte Erstellung von Netzpl\"anen}
\author{Christian Peters}
\institute{Institut f\"ur Kernphysik}
\date{29. August 2018}
\titlegraphic{
  \includegraphics[width=\paperwidth]{placeholder}
}

\begin{document}
\maketitle

\begin{frame}
  \frametitle{Arbeiten im Institut}
  \begin{itemize}
    \item CLAS12 Project
      \begin{itemize}
        \item Teilchendetektor der Thomas Jefferson National Facility,
          Newport News Virginia, U.S.A.
        \item Elektronen werden beschleunigt und zur Kollision
          gebracht
        \item Teilchenbahnen werden durch Driftkammer registriert
      \end{itemize}
    \item Starke Strahlung in der Driftkammer \(\rightarrow\) Defekte
      m\"oglich
    \item Erkennung diser Defekte mithilfe von Algorithmen der
      k\"unstlichen Intelligenz
      \begin{itemize}
        \item Deep Learning, Convolutional Neural Networks
        \item Charakteristische Muster werden gesucht und erkannt
        \item Resultat: \textit{Fault Detector}
      \end{itemize}
  \end{itemize}
\end{frame}

\begin{frame}
  \frametitle{Was ist ein Netzplan?}
  \begin{itemize}
    \item Terminplanung gro{\ss}er Projekte sehr komplex
      \begin{itemize}
        \item Viele einzelne Vorg\"ange, die untereinander
          \textit{vernetzt} sind
        \item Abh\"angigkeiten oft stark verzweigt
        \item Per Hand kaum aufzul\"osen
      \end{itemize}
    \item Wie lange Dauert ein Projekt?
      \begin{itemize}
        \item Welche Vorg\"ange d\"urfen sich nicht verz\"ogern?
        \item Was sind die \textit{kritischen Pfade} durch ein
          Projekt?
        \item Wo kann effektiv Zeit gespart werden?
      \end{itemize}
    \item Technisches Hilfsmittel: Netzplan
      \begin{itemize}
        \item Verkettung aller Vorg\"ange nach ihren Abh\"angigkeiten
        \item Basis f\"ur automatisierte Berechnungen der relevanten
          Gr\"o{\ss}en
      \end{itemize}
    \item Formal: Gerichteter azyklischer Graph
      \begin{itemize}
        \item Vorg\"ange bilden Knoten des Graphen
        \item Abh\"angigkeiten legen die Kanten fest
      \end{itemize}
  \end{itemize}
\end{frame}

\begin{frame}
  \frametitle{Eigenschaften von Vorg\"angen}
  \begin{itemize}
    \item Zu Beginn spezifiziert:
      \begin{itemize}
        \item Eindeutige Vorgangsnummer
        \item Lesbare Vorgangsbezeichnung
        \item Dauer
        \item Vorg\"anger und Nachfolger
      \end{itemize}
    \item Zu berechnen:
      \begin{itemize}
        \item Fr\"uhester und sp\"atester Anfangszeitpunkt (FAZ und SAZ)
        \item Fr\"uhester und sp\"atester Endzeitpunkt (FEZ und SEZ)
        \item Gesamtpuffer
          \begin{itemize}
            \item Spielraum, der das Projektende nicht gef\"ahrdet
          \end{itemize}
        \item Freier Puffer
          \begin{itemize}
            \item Spielraum, der die fr\"uheste Abarbeitung der Nachfolger
              nicht gef\"ahrdet
          \end{itemize}
      \end{itemize}
  \end{itemize}
\end{frame}

\begin{frame}
  \frametitle{Algorithmische Konstruktion eines Netzplans}
  \begin{enumerate}
    \item Einlesen der Vorg\"ange
    \item Initialisierung des Netzplans
      \begin{itemize}
        \item Sind Vorg\"anger und Nachfolger konsistent?
        \item H\"angt der Graph zusammen?
        \item Ist der Graph zyklenfrei?
      \end{itemize}
    \item Vorw\"artsrechnung
      \begin{itemize}
        \item Berechnung von FAZ und FEZ
      \end{itemize}
    \item R\"uckw\"artsrechnung
      \begin{itemize}
        \item Berechnung von SAZ und SEZ
      \end{itemize}
    \item Zeitreserven bestimmen
      \begin{itemize}
        \item Berechnung von GP und FP
      \end{itemize}
  \end{enumerate}
\end{frame}

\begin{frame}
  \frametitle{Pr\"ufen auf Zusammenhang}
  \begin{enumerate}
    \item Erzeuge Adjazenzmatrix $A$ des Graphen
      \begin{itemize}
        \item Gibt an, welche Vorg\"ange wie voneinander abh\"angen
        \item Element $a_{ij}$ ist 1, falls Vorgang $i$ Vorg\"anger
          von Vorgang $j$ ist, ansonsten 0
      \end{itemize}
    \item Erzeuge symmetrische Version $A'$ von $A$
      \begin{itemize}
        \item Richtung der Kanten ist egal f\"ur das Zusammenh\"angen
          des Graphen
        \item Setze $a_{ji}=1$, falls $a_{ij}=1$
      \end{itemize}
    \item Traversiere nun den Graphen auf Basis von $A'$
      \begin{itemize}
        \item Bleiben Knoten \"ubrig, war der Graph nicht zusammenh\"angend!
      \end{itemize}
  \end{enumerate}
\end{frame}

\begin{frame}
  \frametitle{Pr\"ufen auf Zyklen}
  \begin{enumerate}
    \item Erzeuge beginnend bei jedem Startknoten (Knoten ohne
      Vorg\"anger) eine Expansion des Graphen
      \begin{itemize}
        \item D.h. erzeuge sukzessive alle m\"oglichen Pfade durch den Graph
      \end{itemize}
    \item Teste in jedem Schritt der Expansion, ob ein Knoten doppelt
      in einem Pfad vorkommt
      \begin{itemize}
        \item In diesem Fall wurde ein Zyklus erkannt!
      \end{itemize}
  \end{enumerate}
\end{frame}

\begin{frame}
  \frametitle{Vorw\"artsrechnung}
  \begin{itemize}
    \item Berechne FAZ und FEZ aller Vorg\"ange
    \item Aktualisiere die Werte entlang der Abh\"angigkeiten
  \end{itemize}
  \begin{figure}
    \resizebox{\textwidth}{!}{%\sProofOn
\resizebox{\textwidth}{!}{
\begin{struktogramm}(150, 71)
  \assign{Erzeuge Queue \textit{abzuarbeiten}, die zu Beginn nur die Startknoten enth\"alt}
  \while{Solange \textit{abzuarbeiten} nicht leer ist}
  \assign{Entferne erstes Element \textit{aktKnoten}}
  \assign{Setze FEZ von \textit{aktKnoten} auf FAZ+Dauer}
  \while{F\"ur jeden Nachfolger \textit{aktKind} von \textit{aktKnoten}}
  \ifthenelse{3}{3}{\textit{aktKind}.FAZ \textless \textit{aktKnoten}.FEZ}{\sTrue}{\sFalse}
  \assign{Setze FAZ von \textit{aktKind} auf FEZ von \textit{aktKnoten}}
  \assign{F\"uge \textit{aktKind} der Queue \textit{abzuarbeiten} hinten an}
  \change
  \ifend
  \whileend
  \whileend
\end{struktogramm}
}
%\sProofOff
}
  \end{figure}
\end{frame}

\end{document}
