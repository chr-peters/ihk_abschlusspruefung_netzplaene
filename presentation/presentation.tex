\documentclass[ngerman, t]{beamer}

\usepackage[utf8]{inputenc}
\usepackage[T1]{fontenc}
\usepackage[ngerman]{babel}

\usetheme{Juelich}

\title{Automatisierte Erstellung von Netzpl\"anen}
\author{Christian Peters}
\institute{Institut f\"ur Kernphysik}
\date{29. August 2018}
\titlegraphic{
  \includegraphics[width=\paperwidth]{placeholder}
}

\begin{document}
\maketitle

\begin{frame}
  \frametitle{Arbeiten im Institut}
  \begin{itemize}
    \item CLAS12 Project
      \begin{itemize}
        \item Teilchendetektor der Thomas Jefferson National Facility,
          Newport News Virginia, U.S.A.
        \item Elektronen werden beschleunigt und zur Kollision
          gebracht
        \item Teilchenbahnen werden durch Driftkammer registriert
      \end{itemize}
    \item Starke Strahlung in der Driftkammer \(\rightarrow\) Defekte
      m\"oglich
    \item Erkennung diser Defekte mithilfe von Algorithmen der
      k\"unstlichen Intelligenz
      \begin{itemize}
        \item Deep Learning, Convolutional Neural Networks
        \item Charakteristische Muster werden gesucht und erkannt
        \item Resultat: \textit{Fault Detector}
      \end{itemize}
  \end{itemize}
\end{frame}

\end{document}
