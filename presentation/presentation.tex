\documentclass[ngerman, t]{beamer}

\usepackage[utf8]{inputenc}
\usepackage[T1]{fontenc}
\usepackage[ngerman]{babel}
\usepackage[ngerman]{struktex}
\usepackage{pgf-umlcd}

\usetheme{Juelich}

\title{Automatisierte Erstellung von Netzpl\"anen}
\author{Christian Peters}
\institute{Institut f\"ur Kernphysik}
\date{29. August 2018}
\titlegraphic{
  \includegraphics[width=\paperwidth]{placeholder}
}

\begin{document}
\maketitle

\begin{frame}
  \frametitle{Was ist ein Netzplan?}
  \begin{itemize}
    \item Terminplanung gro{\ss}er Projekte sehr komplex
      \begin{itemize}
        \item Viele einzelne Vorg\"ange, die untereinander
          \textit{vernetzt} sind
        \item Abh\"angigkeiten oft stark verzweigt
        \item Per Hand kaum aufzul\"osen
      \end{itemize}
    \item Wie lange Dauert ein Projekt?
      \begin{itemize}
        \item Welche Vorg\"ange d\"urfen sich nicht verz\"ogern?
        \item Was sind die \textit{kritischen Pfade} durch ein
          Projekt?
        \item Wo kann effektiv Zeit gespart werden?
      \end{itemize}
    \item Technisches Hilfsmittel: Netzplan
      \begin{itemize}
        \item Verkettung aller Vorg\"ange nach ihren Abh\"angigkeiten
        \item Basis f\"ur automatisierte Berechnungen der relevanten
          Gr\"o{\ss}en
      \end{itemize}
    \item Formal: Gerichteter azyklischer Graph
      \begin{itemize}
        \item Vorg\"ange bilden Knoten des Graphen
        \item Abh\"angigkeiten legen die Kanten fest
      \end{itemize}
  \end{itemize}
\end{frame}

\begin{frame}
  \frametitle{Algorithmische Konstruktion eines Netzplans}
  \begin{enumerate}
    \item Einlesen der Vorg\"ange
    \item Initialisierung des Netzplans
      \begin{itemize}
        \item Sind Vorg\"anger und Nachfolger konsistent?
        \item H\"angt der Graph zusammen?
        \item Ist der Graph zyklenfrei?
      \end{itemize}
    \item Vorw\"artsrechnung
      \begin{itemize}
        \item Berechnung von FAZ und FEZ
      \end{itemize}
    \item R\"uckw\"artsrechnung
      \begin{itemize}
        \item Berechnung von SAZ und SEZ
      \end{itemize}
    \item Zeitreserven bestimmen
      \begin{itemize}
        \item Berechnung von GP und FP
      \end{itemize}
  \end{enumerate}
\end{frame}

\begin{frame}
  \frametitle{Pr\"ufen auf Zusammenhang}
  \begin{enumerate}
    \item Erzeuge Adjazenzmatrix $A$ des Graphen
      \begin{itemize}
        \item Gibt an, welche Vorg\"ange wie voneinander abh\"angen
        \item Element $a_{ij}$ ist 1, falls Vorgang $i$ Vorg\"anger
          von Vorgang $j$ ist, ansonsten 0
      \end{itemize}
    \item Erzeuge symmetrische Version $A'$ von $A$
      \begin{itemize}
        \item Richtung der Kanten ist egal f\"ur das Zusammenh\"angen
          des Graphen
        \item Setze $a_{ji}=1$, falls $a_{ij}=1$
      \end{itemize}
    \item Traversiere nun den Graphen auf Basis von $A'$
      \begin{itemize}
        \item Bleiben Knoten \"ubrig, war der Graph nicht zusammenh\"angend!
      \end{itemize}
  \end{enumerate}
\end{frame}

\begin{frame}
  \frametitle{Pr\"ufen auf Zyklen}
  \begin{enumerate}
    \item Erzeuge beginnend bei jedem Startknoten (Knoten ohne
      Vorg\"anger) eine Expansion des Graphen
      \begin{itemize}
        \item D.h. erzeuge sukzessive alle m\"oglichen Pfade durch den Graph
      \end{itemize}
    \item Teste in jedem Schritt der Expansion, ob ein Knoten doppelt
      in einem Pfad vorkommt
      \begin{itemize}
        \item In diesem Fall wurde ein Zyklus erkannt!
      \end{itemize}
  \end{enumerate}
\end{frame}

\begin{frame}
  \frametitle{Vorw\"artsrechnung}
  \begin{itemize}
    \item Berechne FAZ und FEZ aller Vorg\"ange
    \item Aktualisiere die Werte entlang der Abh\"angigkeiten
  \end{itemize}
  \begin{figure}
    \resizebox{\textwidth}{!}{%\sProofOn
\resizebox{\textwidth}{!}{
\begin{struktogramm}(150, 71)
  \assign{Erzeuge Queue \textit{abzuarbeiten}, die zu Beginn nur die Startknoten enth\"alt}
  \while{Solange \textit{abzuarbeiten} nicht leer ist}
  \assign{Entferne erstes Element \textit{aktKnoten}}
  \assign{Setze FEZ von \textit{aktKnoten} auf FAZ+Dauer}
  \while{F\"ur jeden Nachfolger \textit{aktKind} von \textit{aktKnoten}}
  \ifthenelse{3}{3}{\textit{aktKind}.FAZ \textless \textit{aktKnoten}.FEZ}{\sTrue}{\sFalse}
  \assign{Setze FAZ von \textit{aktKind} auf FEZ von \textit{aktKnoten}}
  \assign{F\"uge \textit{aktKind} der Queue \textit{abzuarbeiten} hinten an}
  \change
  \ifend
  \whileend
  \whileend
\end{struktogramm}
}
%\sProofOff
}
  \end{figure}
\end{frame}

\begin{frame}
  \frametitle{R\"uckw\"artsrechnung}
  \begin{itemize}
    \item Berechne SAZ und SEZ aller Vorg\"ange
    \item Vorgehensweise analog zur Vorw\"artsrechnung
  \end{itemize}
  \begin{enumerate}
    \item Beginne mit einer Queue aus den Endknoten (Knoten ohne
      Nachfolger)
      \begin{itemize}
        \item Setze bei allen Endknoten $SEZ=FEZ$, da diese keinen
          Puffer haben
      \end{itemize}
    \item Arbeite die Queue wie bei der Vorw\"artsrechnung ab
      \begin{itemize}
        \item Diesmal setze $SAZ=SEZ-Dauer$
        \item Aktualisiere diesmal die $SEZ$ der Vorg\"anger
      \end{itemize}
  \end{enumerate}
\end{frame}

\begin{frame}
  \frametitle{Berechnung der Zeitreserven}
  \begin{enumerate}
    \item Durchlaufe alle Vorg\"ange
    \item Setze $GP=SAZ-FAZ$ (alternativ $GP=SEZ-FEZ$)
    \item Setze $FP=\min\limits_{i \in \{Nachfolger\}} FAZ_i - FEZ$
  \end{enumerate}
  $\rightarrow$ Verwende diese Werte als Grundlage f\"ur die Suche
  nach kritischen Pfaden!
  \begin{itemize}
    \item Ein Vorgang hei{\ss}t \textit{kritisch}, falls $GP=FP=0$
    \item Ein Pfad von Start- zu Endknoten, bestehend nur aus
      kritischen Vorg\"angen, hei{\ss}t \textit{kritischer Pfad}
  \end{itemize}
\end{frame}

\begin{frame}
  \frametitle{Auffinden kritischer Pfade}
  \begin{figure}
    \resizebox{!}{100px}{%\sProofOn
\resizebox{\textwidth}{!}{
\begin{struktogramm}(150, 125)
  \assign{Erzeuge leere Liste \textit{Resultat} f\"ur die kritischen Pfade}
  \while{F\"ur jeden Startknoten \textit{aktStartKnoten}}
  \assign{Erzeuge leere Queue \textit{Pfade}, die abzuarbeitende Pfade enth\"alt}
  \ifthenelse{3}{3}{Ist \textit{aktStartKnoten} kritisch?}{\sTrue}{\sFalse}
  \assign{F\"uge ihn als ``Startpfad'' der Queue \textit{Pfade} hinzu}
  \change
  \exit{Betrachte n\"achsten Startknoten}
  \ifend
  \while{Solange \textit{Pfade} noch Elemente enth\"alt}
  \assign{Entferne erstes Element \textit{aktPfad}}
  \ifthenelse{3}{3}{Ist der letzte Knoten von \textit{aktPfad} ein Endknoten?}{\sTrue}{\sFalse}
  \assign{F\"uge \textit{aktPfad} der Liste \textit{Resultat} hinzu}
  \exit{Continue}
  \change
  \ifend
  \while{F\"ur jeden \textit{kritischen Nachfolger} des letzten Elements aus \textit{aktPfad}}
  \assign{Erzeuge \textit{neuerPfad} = \textit{aktPfad} + \textit{kritischer Nachfolger}}
  \assign{F\"uge \textit{neuerPfad} vorne an \textit{Pfade} an (\(\rightarrow\)Tiefensuche)}
  \whileend
  \whileend
  \whileend
  \assign{return \textit{Resultat}}
\end{struktogramm}
}
%\sProofOff
}
  \end{figure}
\end{frame}

\begin{frame}
  \frametitle{Objektorientierte Realisierung}
  \begin{figure}
    \resizebox{!}{200px}{
      \tikzset{font=\ttfamily}
\renewcommand{\umltextcolor}{black}
\renewcommand{\umldrawcolor}{black}
\renewcommand{\umlfillcolor}{white}
\begin{tikzpicture}%[show background grid]
  \begin{class}[text width=13.5cm]{Vorgang}{0,0}
    \attribute{-nummer: int}
    \attribute{-bezeichnung: String}
    \attribute{-dauer: int}
    \attribute{-vorgaenger: int [0..*]}
    \attribute{-nachfolger: int [0..*]}
    \attribute{-faz: int}
    \attribute{-fez: int}
    \attribute{-saz: int}
    \attribute{-sez: int}
    \attribute{-gp: int}
    \attribute{-fp: int}

    \operation{+Vorgang(nummer: int, bezeichnung: String, dauer: int)}
    \operation{+addVorgaenger(vorgaenger: int)}
    \operation{+addNachfolger(nachfolger: int)}
    \operation{+istKritisch(): boolean}
    \operation{+toString(): String}
  \end{class}
  \begin{class}[text width=11cm]{Netzplan}{-1.25, -10}
    \attribute{-vorgaenge: Vorgang [1..*]}
    \attribute{-adjazenzen: int [n][n]}
    \attribute{-startKnoten: int[1..*]}
    \attribute{-endKnoten: int [1..*]}
    \attribute{-toInternal: Map\textless Integer, Integer\textgreater}
    \attribute{-fromInternal: Map\textless Integer, Integer\textgreater}

    \operation{+Netzplan(vorgaenge: Vorgang [1..*])}
    \operation{+getDauer(): int}
    \operation{+getKritischePfade(): List\textless List\textless Integer\textgreater\textgreater}
    \operation{-erzeugeAdjazenzen()}
    \operation{-istZyklenfrei(): boolean}
    \operation{-istZusammenhaengend(): boolean}
    \operation{-vorwaertsRechnung()}
    \operation{-rueckwaertsRechnung()}
    \operation{-zeitreserven()}
  \end{class}
  \begin{class}[text width=8cm]{VorgangLeser}{12, 0}
    \attribute{-vorgaenge: Vorgang [1..*]}
    \attribute{-ueberschrift: String}

    \operation{+VorgangLeser(datei: String)}
    \operation{+getVorgaenge(): Vorgang [1..*]}
    \operation{+getUeberschrift(): String}
    \operation{-leseVorgaenge(datei: String)}
  \end{class}
  \begin{class}[text width=13cm]{ProjektReport}{11.5, -10}
    \attribute{-datei: String}

    \operation{+ProjektReport(datei: String)}
    \operation{+erzeugeReport(plan: Netzplan, ueberschrift: String)}
  \end{class}

  \aggregation{VorgangLeser}{}{}{Vorgang}
  \aggregation{Netzplan}{}{}{Vorgang}

  %% \begin{interface}[text width=7.5cm]{FieldDescriptor}{-4, -4}
  %%   \operation[0]{getStrength( location: Vector3D ): Vector3D}
  %% \end{interface}
  
  %% \begin{interface}[text width=6.5cm]{Condition}{4, -4}
  %%   \operation[0]{check( track: vector<Particle> ): bool}
  %% \end{interface}

  %% \begin{class}[text width=6cm]{MaximumDistanceCondition}{-4, -7}
  %%   \implement{Condition}
  %%   \attribute{- maxDistance: double}
  %%   \attribute{- referencePoint: Vector3D}
  %% \end{class}

  %% \begin{class}[text width=6cm]{PlaneIntersectionCondition}{4, -7}
  %%   \implement{Condition}
  %%   \attribute{- plane: Plane3D}
  %% \end{class}

  %% \draw[umlcd  style  dashed  line ,->] (Propagator) --node[above , sloped ,
  %%   black]{} (FieldDescriptor);

  %% \draw[umlcd  style  dashed  line ,->] (Propagator) --node[above , sloped ,
  %%   black]{} (Condition);
\end{tikzpicture}

    }
  \end{figure}
\end{frame}

\begin{frame}
  \frametitle{Ausblick}
  \begin{itemize}
    \item Implementierung der Pr\"asentationsschicht
      \begin{itemize}
        \item Bisher nur Logik und Datenhaltung
        \item L\"asst sich leicht auf der bestehenden Architektur
          aufsetzen
      \end{itemize}
    \item Optimierung der Speicherung des Graphen
      \begin{itemize}
        \item Speicherbedarf der Adjazenzmatrix w\"achst quadratisch
          mit der Anzahl der Knoten
        \item Ist aber h\"aufig nur d\"unn besetzt
        \item Nutzung dieser Begebenheit, z.B. durch den Einsatz des
          Compressed Row Storage (CRS) Formats
      \end{itemize}
    \item Unterst\"utzung weiterer Ein- und Ausgabeformate, z.B. XML
      oder JSON
      \begin{itemize}
        \item Kapselung der speziellen Ein- und Ausgabelogik hinter
          einer abstrakten Klasse
        \item Wahl des Formats wird so zur Laufzeit m\"oglich
          (Konfiguration mit konkreter Strategie)
        \item Implementierung durch Einsatz des \textit{Strategy Pattern}
      \end{itemize}
  \end{itemize}
\end{frame}

\begin{frame}

\end{frame}

\begin{frame}
  \frametitle{Eigenschaften von Vorg\"angen}
  \begin{itemize}
    \item Zu Beginn spezifiziert:
      \begin{itemize}
        \item Eindeutige Vorgangsnummer
        \item Lesbare Vorgangsbezeichnung
        \item Dauer
        \item Vorg\"anger und Nachfolger
      \end{itemize}
    \item Zu berechnen:
      \begin{itemize}
        \item Fr\"uhester und sp\"atester Anfangszeitpunkt (FAZ und SAZ)
        \item Fr\"uhester und sp\"atester Endzeitpunkt (FEZ und SEZ)
        \item Gesamtpuffer
          \begin{itemize}
            \item Spielraum, der das Projektende nicht gef\"ahrdet
          \end{itemize}
        \item Freier Puffer
          \begin{itemize}
            \item Spielraum, der die fr\"uheste Abarbeitung der Nachfolger
              nicht gef\"ahrdet
          \end{itemize}
      \end{itemize}
  \end{itemize}
\end{frame}

\begin{frame}
  \frametitle{Einlesen}
  \begin{figure}
    \resizebox{!}{100px}{%\sProofOn
\resizebox{\textwidth}{!}{
\begin{struktogramm}(150, 122)
  \while{Solange noch Zeilen zu lesen sind}
  \ifthenelse{6}{3}{Zeile beginnt mit \texttt{//+}}{\sTrue}{\sFalse}
  \ifthenelse{3}{3}{\"Uberschrift in Ordnung?}{\sTrue}{\sFalse}
  \assign{Setze \"Uberschrift}
  \exit{Neue Zeile}
  \change
  \assign{Fehlermeldung: \"Uberschrift nicht in Ordnung}
  \ifend
  \change
  \ifend
  \ifthenelse{5}{5}{Zeile beginnt mit \texttt{//}}{\sTrue}{\sFalse}
    \exit{Neue Zeile}
    \change
  \ifend
  \assign{Trenne Zeile am Semikolon auf}
  \assign{Validiere die einzelnen Teile}
  \ifthenelse{5}{5}{Validierung erfolgreich?}{\sTrue}{\sFalse}
  \assign{Erzeuge hieraus ein Objekt vom Typ Vorgang}
  \assign{F\"uge den Vorgang zum Resultat hinzu}
  \change
  \assign{Fehlermeldung: Teile nicht in Ordnung}
  \ifend
  \whileend
  \return[8]{Gebe Resultat zur\"uck}
\end{struktogramm}
}
%\sProofOff
}
  \end{figure}
\end{frame}

\begin{frame}
  \frametitle{Pr\"ufen auf Zyklen}
  \begin{figure}
    \resizebox{!}{100px}{%\sProofOn
\resizebox{\textwidth}{!}{
\begin{struktogramm}(150, 124)
  \while{F\"ur jeden Startknoten \textit{aktStartKnoten}}
  \assign{Erzeuge leere Queue \textit{Pfade}, die alle abzuarbeitenden Pfade enth\"alt}
  \assign{F\"uge \textit{aktStartKnoten} der Queue \textit{Pfade} als Startpfad hinzu}
  \while{Solange \textit{Pfade} noch Elemente enth\"alt}
  \assign{Entferne erstes Element \textit{aktPfad}}
  \assign{Betrachte letztes Element \textit{aktKnoten} von \textit{aktPfad}}
  \ifthenelse{3}{3}{Ist \textit{aktKnoten} ein Endknoten?}{\sTrue}{\sFalse}
  \exit{\textit{aktPfad} ist zyklenfrei, bearbeite n\"achstes Element von \textit{Pfade}}
  \change
  \ifend
  \while{F\"ur jeden Nachfolger \textit{aktKind} von \textit{aktKnoten}}
  \ifthenelse{3}{3}{Ist \textit{aktKind} schon in \textit{aktPfad} enthalten?}{\sTrue}{\sFalse}
  \assign{Es wurde ein Zyklus erkannt!}
  \change
  \assign{Erzeuge neuen Pfad \textit{aktPfad} + \textit{aktKind}}
  \assign{F\"uge den neuen Pfad hinten zu \textit{Pfade} hinzu}
  \ifend
  \whileend
  \whileend
  \whileend
  \assign{Kein Zyklus gefunden!}
\end{struktogramm}
}
%\sProofOff
}
  \end{figure}
\end{frame}

\begin{frame}
  \frametitle{Arbeiten im Institut}
  \begin{itemize}
    \item CLAS12 Project
      \begin{itemize}
        \item Teilchendetektor der Thomas Jefferson National Facility,
          Newport News Virginia, U.S.A.
        \item Elektronen werden beschleunigt und zur Kollision
          gebracht
        \item Teilchenbahnen werden durch Driftkammer registriert
      \end{itemize}
    \item Starke Strahlung in der Driftkammer \(\rightarrow\) Defekte
      m\"oglich
    \item Erkennung diser Defekte mithilfe von Algorithmen der
      k\"unstlichen Intelligenz
      \begin{itemize}
        \item Deep Learning, Convolutional Neural Networks
        \item Charakteristische Muster werden gesucht und erkannt
        \item Resultat: \textit{Fault Detector}
      \end{itemize}
  \end{itemize}
\end{frame}

\end{document}
