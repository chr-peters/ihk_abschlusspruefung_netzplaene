\chapter{Benutzeranleitung}
\label{Benutzeranleitung}

\section{Lieferumfang}

Im Lieferumfang dieser Software sind folgende Elemente als Inhalt der
.zip-Datei enthalten:
\begin{itemize}
  \item \textbf{Dokumentation:} Enth\"alt die vollst\"andige
    Beschreibung des Softwaresystems sowie der unterliegenden
    Verfahren und ist in
    der Datei \texttt{Dokumentation.pdf} zu finden.
  \item \textbf{Das ausf\"uhrbare Programm:} Das Programm in Form
    der ausf\"uhrbaren .jar-Datei
    \texttt{netzplanerstellung-1.0.0.jar}.
  \item \textbf{Quelltext des Programms:} Der vollst\"andige Quelltext
    des Programms findet sich im Ordner \texttt{src}.
  \item \textbf{Testf\"alle:} Die vollst\"andige Sammlung aller
    Testf\"alle liegt im Ordner \texttt{testcases}.
  \item \textbf{Skript zum Ausf\"uhren aller Testf\"alle:} Liegt als
    Datei \texttt{run\_testcases.ksh} vor.
  \item \textbf{Ant-Buildfile zum automatisierten \"Ubersetzen des
    Quellcodes:} \texttt{build.xml}-Datei.
\end{itemize}

\section{Ausf\"uhren des Programms}

Die Ausf\"uhrung des Programms kann nach dem folgenden Schema eines
Kommandos erfolgen:
\begin{verbatim}
java -jar netzplanerstellung-1.0.0.jar <Eingabedatei> <Ausgabedatei>
\end{verbatim}
Hierbei gibt \texttt{<Eingabedatei>} den Pfad zu einer g\"ultigen
Eingabedatei nach dem in Abschnitt \ref{eingabeformat} beschriebenen
Format an. Wurde das Format nicht eingehalten, so beendet sich das
Programm unter Ausgabe einer Fehlermeldung.
\texttt{<Ausgabedatei>} gibt den Namen der Datei an, in welcher das
Programm die Ergebnisse abspeichern soll.

\textbf{Achtung:} Diese
Datei wird nur erzeugt, wenn sich das Programm fehlerfrei beendet. Im
Falle eines Fehlers wird eine entsprechende Meldung auf der
Kommandozeile ausgegeben, die n\"ahere Informationen enth\"alt.

\section{Ausf\"uhren aller Testf\"alle}

Um alle Testf\"alle hintereinander ausf\"uhren zu k\"onnen, wurde dem
Lieferumfang das Skript \texttt{run\_testcases.ksh} beigelegt, welches
mithilfe eines \textit{Kornshell-Interpreters} ausgef\"uhrt werden
kann.
Der Aufruf kann folgenderma{\ss}en erfolgen:
\begin{verbatim}
./run_testcases.ksh <Programm> <Eingabeordner> <Ausgabeordner>
\end{verbatim}
Hierbei gibt \texttt{<Programm>} den Pfad zur ausf\"uhrbaren .jar-Datei
an, \texttt{<Eingabeordner>} beschreibt, in welchem Ordner die
Testf\"alle zu finden sind und \texttt{<Ausgabeordner>} gibt den
Ordner an, in dem die Ergebnisdateien abgelegt werden. Optional
k\"onnen die Argumente von rechts nach links weggelassen werden. Die
Standardwerte sind in diesem Fall:
\begin{itemize}
  \item \texttt{<Programm> = netzplanerstellung-1.0.0.jar}
  \item \texttt{<Eingabeordner> = testcases}
  \item \texttt{<Ausgabeordner> = testcases}
\end{itemize}
Ein Aufruf von \texttt{./run\_testcases.ksh} arbeitet also mithilfe
dieser Standardwerte.

Die Vorgehensweise dieses Skriptes kann wie folgt beschrieben werden:
Es werden zun\"achst alle Eingabedaten mit der Endung \texttt{.in} aus
dem Ordner \texttt{<Eingabeordner>} erfasst und nacheinander dem
Programm zugef\"uhrt. Zugeh\"orig zu jeder Eingabedatei, die das
Programm fehlerfrei durchlaufen konnte, wird anschlie{\ss}end im
Ordner \texttt{<Ausgabeordner>} eine Ergebnisdatei mit der Endung
\texttt{.out} erzeugt. Konnte die Ergebnisdatei nicht erzeugt werden,
so wird die Fehlermeldung des Programms auf der Kommandozeile
ausgegeben.
