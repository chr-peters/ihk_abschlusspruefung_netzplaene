\chapter{Testdokumentation}
\label{Testdokumentation}

In diesem Kapitel werden nun alle mitgelieferten Testf\"alle
aufgelistet, beschrieben und diskutiert. Die Zusammenstellung dieser
Testf\"alle ist schon w\"ahrend der Implementierung nach dem
\textit{Whitebox}-Prinzip erfolgt: Jede neue Verzweigung im Programm
wurde sofort mithilfe eines neuen Testfalls festgehalten. Um diesen
Vorgang der Entwicklung zu beschreiben, passt ebenfalls der Begriff
\textit{Test-Driven-Development}.

Die Gesamtheit der Testf\"alle wird hier in folgende Kategorieren
unterteilt:
\begin{itemize}
  \item \textbf{Normalf\"alle:} F\"uhren zu einer ordnungsgem\"a{\ss}en
    Abarbeitung des Programms.
  \item \textbf{Sonderf\"alle:} Enthalten Spezialf\"alle, wie z.B. ein
    Minimalbeispiel, um auch ungew\"ohnliche Situationen testen zu
    k\"onnen.
  \item \textbf{Fehlerf\"alle:} F\"uhren nicht zu einer
    ordnungsgem\"a{\ss}en Abarbeitung des Programms und ziehen
    Fehlermeldungen nach sich.
\end{itemize}
S\"amtliche Testf\"alle sind im Ordner \texttt{testcases} zu finden.
Zu jedem Testfall, welcher in einer ordnungsgem\"a{\ss}en Abarbeitung
des Programms m\"undet, findet sich ebenfalls im Ordner
\texttt{testcases} eine zugeh\"orige Ausgabedatei mit der Endung
\texttt{.out}.

\section{Normalf\"alle}
\begin{tabularx}{\textwidth}{l X}
  Name: & Installation von POI Kiosken \\
  Dateiname: & \texttt{ihk\_01.in} \\
  Beschreibung: & Projekt, welches sich mit der Einrichtung eines
  Kiosk-Terminals besch\"aftigt. Es enth\"alt eine Verzweigung nach
  dem Startvorgang, die an Vorgang Nr. 6 wieder zusammengef\"uhrt
  wird. Das Projekt enth\"alt einen einzigen kritischen Pfad. \\
  Diskussion: & Bei diesem Testfall handelt es sich um ein einfaches
  Normalbeispiel, welches die grunds\"atzliche
  Funktionsf\"ahigkeit des Programms unter Beweis stellt. \\
\end{tabularx}
\hrule
\begin{tabularx}{\textwidth}{l X}
  Name: & Wasserfallmodell \\
  Dateiname: & \texttt{ihk\_02\_korrigiert.in} \\
  Beschreibung: & Dieser Testfall veranschaulicht die Vorgehensweise
  bei der Softwareentwicklung nach dem Wasserfallmodell. Es gibt
  keinerlei Verzweigungen und nur einen kritischen Pfad. Dieses
  Beispiel wurde aus der urspr\"unglichen Datei \texttt{ihk\_02.in}
  erzeugt, indem in Zeile 11 der Vorg\"anger von Vorgang 6 auf Vorgang
  5 korrigiert wurde.\\
  Diskussion: & Der Testfall gibt einen simplen Fall ohne
  Verzweigungen an und stellt sicher, dass das Programm auch bei
  listen\"ahnlichen Netzpl\"anen den einzigen kritischen Pfad
  findet.\\
\end{tabularx}
\hrule
\begin{tabularx}{\textwidth}{l X}
  Name: & Beispiel 3 \\
  Dateiname: & \texttt{ihk\_03.in} \\
  Beschreibung: & Dieser Testfall stellt ein Projekt mit mehreren
  Start- und Endvorg\"angen dar. Er enth\"alt mehrere Verzweigungen
  und zwei kritische Pfade.\\
  Diskussion: & Dieser Testfall zeigt, dass das Programm auch in der
  Lage ist, bei mehreren Start- und Endvorg\"angen korrekt zu
  arbeiten. Weiterhin wird demonstriert, dass auch mehrere kritische
  Pfade fehlerfrei gefunden werden.\footnote{Die Ausgabe des Programms
  stimmt an einer Stelle aus gutem Grund nicht mit der aus der
  Aufgabenstellung gegebenen Kontrolll\"osung \"uberein: Der SEZ von
  Vorgang 8 betr\"agt richtigerweise 12 und nicht 13, wie es in der
  Beispielausgabe steht.}\\
\end{tabularx}
\begin{tabularx}{\textwidth}{l X}
  Name: & Beispiel 5 IT-Installation \\
  Dateiname: & \texttt{ihk\_05\_korrigiert.in} \\
  Beschreibung: & Hier wird ein komplexes Beispiel mit 17
  verschiedenen und verzweigten Vorg\"angen illustriert. Dieses
  Beispiel wurde aus der urspr\"unglichen Datei \texttt{ihk\_05.in}
  erzeugt, indem in Zeile 13 bei Vorgang 8 der Vorg\"anger auf Vorgang 4
  korrigiert wurde. Bei diesem Testfall gibt es einen kritischen Pfad.\\
  Diskussion: & Dies ist der bis hierhin komplexeste Testfall, den das
  Programm bisher abarbeiten musste und es zeigt sich, dass es auch
  hier nicht auf Probleme st\"o{\ss}t.\footnote{Aufgrund der Korrektur
  stimmt auch hier die Ausgabe des Programms nicht mit der gegebenen
  Kontrolll\"osung \"uberein.}\\
\end{tabularx}

\section{Sonderf\"alle}
\begin{tabularx}{\textwidth}{l X}
  Name: & Nummern ausser der Reihe \\
  Dateiname: & \texttt{sonderfall\_01.in} \\
  Beschreibung: & Dieser Testfall hat im Wesentlichen den gleichen
  Inhalt wie die Datei \texttt{ihk\_01.in}, bis auf einen Unterschied:
  Die Nummer von ehemals Vorgang 1 wurde auf 700 gesetzt und die
  Nummer von ehemals Vorgang 4 wurde auf -25 festgelegt.\\
  Diskussion: & Hier soll gepr\"uft werden, ob das Programm auch dann
  noch korrekt arbeitet, wenn die Vorgangsnummern nicht genau in einer
  Reihe aufeinander folgen. Dank der Definition der Abbildung \(g\),
  auch als \texttt{toInternal} im Quelltext zu finden
  (Abbildung der externen auf die internen Vorgangsnummern), sind auch
  unorthodoxe Vorgangsnummern kein Problem f\"ur das Programm.\\
\end{tabularx}
\hrule
\begin{tabularx}{\textwidth}{l X}
  Name: & Ein einziger Knoten \\
  Dateiname: & \texttt{sonderfall\_02.in} \\
  Beschreibung: & Dieses Projekt besteht aus nur einem Vorgang,
  welcher gleichzeitig Start- und Endvorgang ist.\\
  Diskussion: & Dieser Testfall stellt das kleinste noch
  funktionsf\"ahige Beispiel dar. Auch hier ist das Programm
  erfolgreich und berechnet den einzigen kritischen Pfad korrekt.\\
\end{tabularx}

\section{Fehlerf\"alle}
\begin{tabularx}{\textwidth}{l X}
  Name: & Wasserfallmodell \\
  Dateiname: & \texttt{ihk\_02.in} \\
  Beschreibung: & Diese Eingabedatei enth\"alt einen Tippfehler:
  Vorgang 6 hat den nicht existenten Vorgang 7 als Vorg\"anger, obwohl
  er eigentlich Nachfolger von Vorgang 5 ist.\\
  Diskussion: & Hier wird erstmals \"uberpr\"uft, ob das Programm
  Inkonsistenzen in der Vorg\"anger - Nachfolger - Relation korrekt
  erkennt.\\
  Fehlermeldung: & \texttt{Fehler bei der Erstellung des Netzplans:
    Inkonsistente Beziehung gefunden! Vorgang 5 hat Vorgang 6 als
    Nachfolger, ist aber selbst nicht Vorgänger von diesem!} \\
\end{tabularx}
\hrule
\begin{tabularx}{\textwidth}{l X}
  Name: & Beispiel 4 mit Zyklus \\
  Dateiname: & \texttt{ihk\_04.in} \\
  Beschreibung: &  Weiterer Test zur Erkennung von Inkonsistenzen.\\
  Diskussion: & Dieser gegebene Fehlerfall sollte vermutlich einen
  Zyklus illustrieren, enth\"alt allerdings Inkonsistenzen in den
  Beziehungen der Vorg\"ange. \\
  Fehlermeldung: & \texttt{Fehler bei der Erstellung des Netzplans:
    Inkonsistente Beziehung gefunden! Vorgang 4 hat Vorgang 3 als
    Nachfolger, ist aber selbst nicht Vorgänger von diesem!} \\
\end{tabularx}
\hrule
\begin{tabularx}{\textwidth}{l X}
  Name: & Beispiel 4 mit Zyklus (korrigiert) \\
  Dateiname: & \texttt{ihk\_04\_korrigiert.in} \\
  Beschreibung: & Das Projekt zu diesem Testfall f\"uhrt nicht zu
  einem g\"ultigen Netzplan, da es einen Zyklus enth\"alt. Die
  Korrektur zum vorigen Beispiel wurde in Zeile 8 durchgef\"uhrt:
  Vorgang 3 hat nun Vorgang 4 auch als Vorg\"anger, damit ein Zyklus
  zustande kommt. \\
  Diskussion: & Hier wird erstmals \"uberpr\"uft, ob das Programm
  Zyklen im Netzplan erkennt. \\
  Fehlermeldung: & \texttt{Fehler bei der Erstellung des Netzplans: Es
    wurde ein Zyklus erkannt! 3->4->3} \\
\end{tabularx}
