\chapter{Testdokumentation}
\label{Testdokumentation}

In diesem Kapitel werden nun alle mitgelieferten Testf\"alle
aufgelistet, beschrieben und diskutiert. Die Zusammenstellung dieser
Testf\"alle ist schon w\"ahrend der Implementierung nach dem
\textit{Whitebox}-Prinzip erfolgt: Jede neue Verzweigung im Programm
wurde sofort mithilfe eines neuen Testfalls festgehalten. Um diesen
Vorgang der Entwicklung zu beschreiben, passt ebenfalls der Begriff
\textit{Test-Driven-Development}.

Die Gesamtheit der Testf\"alle wird hier in folgende Kategorieren
unterteilt:
\begin{itemize}
  \item \textbf{Normalf\"alle:} F\"uhren zu einer ordnungsgem\"a{\ss}en
    Abarbeitung des Programms.
  \item \textbf{Sonderf\"alle:} Enthalten Spezialf\"alle, wie z.B. ein
    Minimalbeispiel, um auch ungew\"ohnliche Situationen testen zu
    k\"onnen.
  \item \textbf{Fehlerf\"alle:} F\"uhren nicht zu einer
    ordnungsgem\"a{\ss}en Abarbeitung des Programms und ziehen
    Fehlermeldungen nach sich.
\end{itemize}
S\"amtliche Testf\"alle sind im Ordner \texttt{testcases} zu finden.
Zu jedem Testfall, welcher in einer ordnungsgem\"a{\ss}en Abarbeitung
des Programms m\"undet, findet sich ebenfalls im Ordner
\texttt{testcases} eine zugeh\"orige Ausgabedatei mit der Endung
\texttt{.out}.

\section{Normalf\"alle}
\begin{tabularx}{\textwidth}{l X}
  Name: & Installation von POI Kiosken \\
  Dateiname: & \texttt{ihk\_01.in} \\
  Beschreibung: & Projekt, welches sich mit der Einrichtung eines
  Kiosk-Terminals besch\"aftigt. Es enth\"alt eine Verzweigung nach
  dem Startvorgang, die an Vorgang Nr. 6 wieder zusammengef\"uhrt
  wird. Das Projekt enth\"alt einen einzigen kritischen Pfad. \\
  Diskussion: & Bei diesem Testfall handelt es sich um ein einfaches
  Normalbeispiel, welches die grunds\"atzliche
  Funktionsf\"ahigkeit des Programms unter Beweis stellt. \\
\end{tabularx}
\hrule
\begin{tabularx}{\textwidth}{l X}
  Name: & Wasserfallmodell \\
  Dateiname: & \texttt{ihk\_02\_korrigiert.in} \\
  Beschreibung: & Dieser Testfall veranschaulicht die Vorgehensweise
  bei der Softwareentwicklung nach dem Wasserfallmodell. Es gibt
  keinerlei Verzweigungen und nur einen kritischen Pfad. Dieses
  Beispiel wurde aus der urspr\"unglichen Datei \texttt{ihk\_02.in}
  erzeugt, indem in Zeile 11 der Vorg\"anger von Vorgang 6 auf Vorgang
  5 korrigiert wurde.\\
  Diskussion: & Der Testfall gibt einen simplen Fall ohne
  Verzweigungen an und stellt sicher, dass das Programm auch bei
  listen\"ahnlichen Netzpl\"anen den einzigen kritischen Pfad
  findet.\\
\end{tabularx}
\hrule
\begin{tabularx}{\textwidth}{l X}
  Name: & Beispiel 3 \\
  Dateiname: & \texttt{ihk\_03.in} \\
  Beschreibung: & Dieser Testfall stellt ein Projekt mit mehreren
  Start- und Endvorg\"angen dar. Er enth\"alt mehrere Verzweigungen
  und zwei kritische Pfade.\\
  Diskussion: & Dieser Testfall zeigt, dass das Programm auch in der
  Lage ist, bei mehreren Start- und Endvorg\"angen korrekt zu
  arbeiten. Weiterhin wird demonstriert, dass auch mehrere kritische
  Pfade fehlerfrei gefunden werden.\footnote{Die Ausgabe des Programms
  stimmt an einer Stelle aus gutem Grund nicht mit der aus der
  Aufgabenstellung gegebenen Kontrolll\"osung \"uberein: Der SEZ von
  Vorgang 8 betr\"agt richtigerweise 12 und nicht 13, wie es in der
  Beispielausgabe steht.}\\
\end{tabularx}
\begin{tabularx}{\textwidth}{l X}
  Name: & Beispiel 5 IT-Installation \\
  Dateiname: & \texttt{ihk\_05\_korrigiert.in} \\
  Beschreibung: & Hier wird ein komplexes Beispiel mit 17
  verschiedenen und verzweigten Vorg\"angen illustriert. Dieses
  Beispiel wurde aus der urspr\"unglichen Datei \texttt{ihk\_05.in}
  erzeugt, indem in Zeile 13 bei Vorgang 8 der Vorg\"anger auf Vorgang 4
  korrigiert wurde. Bei diesem Testfall gibt es einen kritischen Pfad.\\
  Diskussion: & Dies ist der bis hierhin komplexeste Testfall, den das
  Programm bisher abarbeiten musste und es zeigt sich, dass es auch
  hier nicht auf Probleme st\"o{\ss}t.\footnote{Aufgrund der Korrektur
  stimmt auch hier die Ausgabe des Programms nicht mit der gegebenen
  Kontrolll\"osung \"uberein.}\\
\end{tabularx}

\section{Sonderf\"alle}
\begin{tabularx}{\textwidth}{l X}
  Name: & Nummern ausser der Reihe \\
  Dateiname: & \texttt{sonderfall\_01.in} \\
  Beschreibung: & Dieser Testfall hat im Wesentlichen den gleichen
  Inhalt wie die Datei \texttt{ihk\_01.in}, bis auf einen Unterschied:
  Die Nummer von ehemals Vorgang 1 wurde auf 700 gesetzt und die
  Nummer von ehemals Vorgang 4 wurde auf -25 festgelegt.\\
  Diskussion: & Hier soll gepr\"uft werden, ob das Programm auch dann
  noch korrekt arbeitet, wenn die Vorgangsnummern nicht genau in einer
  Reihe aufeinander folgen. Dank der Definition der Abbildung \(g\),
  auch als \texttt{toInternal} im Quelltext zu finden
  (Abbildung der externen auf die internen Vorgangsnummern), sind auch
  unorthodoxe Vorgangsnummern kein Problem f\"ur das Programm.\\
\end{tabularx}
\hrule
\begin{tabularx}{\textwidth}{l X}
  Name: & Ein einziger Knoten \\
  Dateiname: & \texttt{sonderfall\_02.in} \\
  Beschreibung: & Dieses Projekt besteht aus nur einem Vorgang,
  welcher gleichzeitig Start- und Endvorgang ist.\\
  Diskussion: & Dieser Testfall stellt das kleinste noch
  funktionsf\"ahige Beispiel dar. Auch hier ist das Programm
  erfolgreich und berechnet den einzigen kritischen Pfad korrekt.\\
\end{tabularx}

\section{Fehlerf\"alle}
\begin{tabularx}{\textwidth}{l X}
  Name: & Wasserfallmodell \\
  Dateiname: & \texttt{ihk\_02.in} \\
  Beschreibung: & Diese Eingabedatei enth\"alt einen Tippfehler:
  Vorgang 6 hat den nicht existenten Vorgang 7 als Vorg\"anger, obwohl
  er eigentlich Nachfolger von Vorgang 5 ist.\\
  Diskussion: & Hier wird erstmals \"uberpr\"uft, ob das Programm
  Inkonsistenzen in der Vorg\"anger - Nachfolger - Relation korrekt
  erkennt.\\
  Fehlermeldung: & \texttt{Fehler bei der Erstellung des Netzplans:
    Inkonsistente Beziehung gefunden! Vorgang 5 hat Vorgang 6 als
    Nachfolger, ist aber selbst nicht Vorgänger von diesem!} \\
\end{tabularx}
\hrule
\begin{tabularx}{\textwidth}{l X}
  Name: & Beispiel 4 mit Zyklus \\
  Dateiname: & \texttt{ihk\_04.in} \\
  Beschreibung: &  Weiterer Test zur Erkennung von Inkonsistenzen.\\
  Diskussion: & Dieser gegebene Fehlerfall sollte vermutlich einen
  Zyklus illustrieren, enth\"alt allerdings Inkonsistenzen in den
  Beziehungen der Vorg\"ange. \\
  Fehlermeldung: & \texttt{Fehler bei der Erstellung des Netzplans:
    Inkonsistente Beziehung gefunden! Vorgang 4 hat Vorgang 3 als
    Nachfolger, ist aber selbst nicht Vorgänger von diesem!} \\
\end{tabularx}
\hrule
\begin{tabularx}{\textwidth}{l X}
  Name: & Beispiel 4 mit Zyklus (korrigiert) \\
  Dateiname: & \texttt{ihk\_04\_korrigiert.in} \\
  Beschreibung: & Das Projekt zu diesem Testfall f\"uhrt nicht zu
  einem g\"ultigen Netzplan, da es einen Zyklus enth\"alt. Die
  Korrektur zum vorigen Beispiel wurde in Zeile 8 durchgef\"uhrt:
  Vorgang 3 hat nun Vorgang 4 auch als Vorg\"anger, damit ein Zyklus
  zustande kommt. \\
  Diskussion: & Hier wird erstmals \"uberpr\"uft, ob das Programm
  Zyklen im Netzplan erkennt. \\
  Fehlermeldung: & \texttt{Fehler bei der Erstellung des Netzplans: Es
    wurde ein Zyklus erkannt! 3->4->3} \\
\end{tabularx}
\begin{tabularx}{\textwidth}{l X}
  Name: & Nicht existierender Vorg\"anger \\
  Dateiname: & \texttt{fehlerfall\_01.in} \\
  Beschreibung: & Dieser Fehlerfall basiert auf der g\"ultigen Datei
  \texttt{ihk\_01.in}, nur dass in Zeile 7 der nicht existierende
  Vorgang 8 als Vorg\"anger von Vorgang 2 eingetragen wurde.\\
  Diskussion: & Dieser Testfall zeigt, dass das Programm in der Lage
  ist, nicht existierende Vorg\"anger zu erkennen. \\
  Fehlermeldung: & \texttt{Fehler bei der Erstellung des Netzplans:
    Vorgang 2 hat Vorgang 8 als Vorgänger, obwohl dieser nicht
    existiert!} \\
\end{tabularx}
\hrule
\begin{tabularx}{\textwidth}{l X}
  Name: & Leere \"Uberschrift \\
  Dateiname: & \texttt{fehlerfall\_02.in} \\
  Beschreibung: & Dieser Fehlerfall basiert auf der g\"ultigen Datei
  \texttt{ihk\_01.in}, nur dass in Zeile 2 eine leere \"Uberschrift
  angegeben wurde. \\
  Diskussion: & Nach Abschnitt \ref{eingabeformat} muss sich in jeder
  g\"ultigen Eingabedatei genau eine nicht-leere \"Uberschrift
  befinden. Das Programm erkennt diese Datei also korrekt als
  ung\"ultig an. \\
  Fehlermeldung: & \texttt{Fehler beim Einlesen in Zeile 2: Leere
    Überschriften sind ungültig!} \\
\end{tabularx}
\hrule
\begin{tabularx}{\textwidth}{l X}
  Name: & Keine \"Uberschrift \\
  Dateiname: & \texttt{fehlerfall\_03.in} \\
  Beschreibung: & Dieser Fehlerfall basiert auf der g\"ultigen Datei
  \texttt{ihk\_01.in}, nur dass keine \"Uberschrift
  angegeben wurde. \\
  Diskussion: & Nach Abschnitt \ref{eingabeformat} muss sich in jeder
  g\"ultigen Eingabedatei genau eine nicht-leere \"Uberschrift
  befinden. Das Programm erkennt diese Datei also korrekt als
  ung\"ultig an. \\
  Fehlermeldung: & \texttt{Fehler beim Einlesen: Keine Überschrift
    gefunden!} \\
\end{tabularx}
\hrule
\begin{tabularx}{\textwidth}{l X}
  Name: & Zwei \"Uberschriften \\
  Dateiname: & \texttt{fehlerfall\_04.in} \\
  Beschreibung: & Dieser Fehlerfall basiert auf der g\"ultigen Datei
  \texttt{ihk\_01.in}, nur dass in Zeile 3 eine zweite \"Uberschrift
  angegeben wurde. \\
  Diskussion: & Nach Abschnitt \ref{eingabeformat} muss sich in jeder
  g\"ultigen Eingabedatei genau eine nicht-leere \"Uberschrift
  befinden. Das Programm erkennt diese Datei also korrekt als
  ung\"ultig an. \\
  Fehlermeldung: & \texttt{Fehler beim Einlesen in Zeile 3: Nur eine
    Überschrift pro Datei erlaubt!} \\
\end{tabularx}
\begin{tabularx}{\textwidth}{l X}
  Name: & Ung\"ultige Anzahl Spalten \\
  Dateiname: & \texttt{fehlerfall\_05.in} \\
  Beschreibung: & Dieser Fehlerfall basiert auf der g\"ultigen Datei
  \texttt{ihk\_01.in}, nur dass in Zeile 6 die Spalte Dauer entfernt
  wurde. \\
  Diskussion: & Eine g\"ultige Datenzeile besteht nach Abschnit
  \ref{eingabeformat} immer aus genau 5 durch Semikola getrennten
  Elementen. Da hier ein Element fehlt, wird dies zurecht als Fehler
  erkannt.\\
  Fehlermeldung: & \texttt{Fehler beim Einlesen in Zeile 6: Erwarte 5
    Elemente pro Zeile, 4 erhalten.} \\
\end{tabularx}
\hrule
\begin{tabularx}{\textwidth}{l X}
  Name: & Ung\"ultige Vorgangsnummer \\
  Dateiname: & \texttt{fehlerfall\_06.in} \\
  Beschreibung: & Dieser Fehlerfall basiert auf der g\"ultigen Datei
  \texttt{ihk\_01.in}, nur dass in Zeile 11 die ung\"ultige
  Vorgangsnummer ``6a'' eingetragen wurde. \\
  Diskussion: & G\"ultige Vorgangsnummern sind nach Abschnitt
  \ref{vorgang} immer ganze Zahlen. Das Programm erkennt hier also
  zurecht eine ung\"ultige Vorgangsnummer und beendet sich mit einem
  Fehler.\\
  Fehlermeldung: & \texttt{Fehler beim Einlesen in Zeile 11: Ungültige
    Vorgangsnummer 6a} \\
\end{tabularx}
\hrule
\begin{tabularx}{\textwidth}{l X}
  Name: & Leere Vorgangsbezeichnung \\
  Dateiname: & \texttt{fehlerfall\_07.in} \\
  Beschreibung: & Dieser Fehlerfall basiert auf der g\"ultigen Datei
  \texttt{ihk\_01.in}, nur dass in Zeile 10 eine leere
  Vorgangsbezeichnung eingetragen wurde. \\
  Diskussion: & G\"ultige Vorgangsbezeichnungen sind nach Abschnitt
  \ref{vorgang} immer nicht-leere Zeichenketten. Das Programm erkennt
  hier also zurecht eine ung\"ultige Vorgangsbezeichnung und beendet
  sich mit einem Fehler.\\
  Fehlermeldung: & \texttt{Fehler beim Einlesen in Zeile 10: Keine
    leere Vorgangsbezeichnung erlaubt!} \\
\end{tabularx}
\hrule
\begin{tabularx}{\textwidth}{l X}
  Name: & Ung\"ultige Dauer \\
  Dateiname: & \texttt{fehlerfall\_08.in} \\
  Beschreibung: & Dieser Fehlerfall basiert auf der g\"ultigen Datei
  \texttt{ihk\_01.in}, nur dass in Zeile 9 die ung\"ultige Dauer
  ``6a'' eingetragen wurde. \\
  Diskussion: & G\"ultige Dauern sind nach Abschnitt
  \ref{vorgang} immer Elemente der nat\"urlichen Zahlen. Das Programm
  erkennt hier also zurecht eine ung\"ultige Dauer und beendet
  sich mit einem Fehler.\\
  Fehlermeldung: & \texttt{Fehler beim Einlesen in Zeile 9: Ungültige Vorgangsdauer  6a} \\
\end{tabularx}
\begin{tabularx}{\textwidth}{l X}
  Name: & Negative Dauer \\
  Dateiname: & \texttt{fehlerfall\_09.in} \\
  Beschreibung: & Dieser Fehlerfall basiert auf der g\"ultigen Datei
  \texttt{ihk\_01.in}, nur dass in Zeile 9 eine negative Dauer
  eingetragen wurde. \\
  Diskussion: & G\"ultige Dauern sind nach Abschnitt
  \ref{vorgang} immer Elemente der nat\"urlichen Zahlen. Das Programm
  erkennt hier also zurecht eine ung\"ultige Dauer und beendet
  sich mit einem Fehler.\\
  Fehlermeldung: & \texttt{Fehler beim Einlesen in Zeile 9: Dauern <=
    0 sind nicht erlaubt!} \\
\end{tabularx}
\hrule
\begin{tabularx}{\textwidth}{l X}
  Name: & Keine Dauer \\
  Dateiname: & \texttt{fehlerfall\_10.in} \\
  Beschreibung: & Dieser Fehlerfall basiert auf der g\"ultigen Datei
  \texttt{ihk\_01.in}, nur dass in Zeile 9 keine Dauer
  eingetragen wurde. \\
  Diskussion: & G\"ultige Dauern sind nach Abschnitt
  \ref{vorgang} immer Elemente der nat\"urlichen Zahlen. Das Programm
  erkennt hier also zurecht eine ung\"ultige Dauer und beendet
  sich mit einem Fehler.\\
  Fehlermeldung: & \texttt{Fehler beim Einlesen in Zeile 9: Ungültige Vorgangsdauer} \\
\end{tabularx}
\hrule
\begin{tabularx}{\textwidth}{l X}
  Name: & Ung\"ultige Vorg\"anger \\
  Dateiname: & \texttt{fehlerfall\_11.in} \\
  Beschreibung: & Dieser Fehlerfall basiert auf der g\"ultigen Datei
  \texttt{ihk\_01.in}, nur dass in Zeile 8 ein ung\"ultiger
  Vorg\"anger, n\"amlich die Zeichenkette ``falsch'',
  eingetragen wurde. \\
  Diskussion: & G\"ultige Vorg\"anger sind nach Abschnitt
  \ref{vorgang} immer g\"ultige Vorgangsnummern, die durch ein Komma
  getrennt wurden. Das Programm erkennt hier also zurecht einen
  ung\"ultigen Vorg\"anger und beendet sich mit einem Fehler.\\
  Fehlermeldung: & \texttt{Fehler beim Einlesen in Zeile 8: Ungültiger
    Vorgänger falsch} \\
\end{tabularx}
\hrule
\begin{tabularx}{\textwidth}{l X}
  Name: & Doppelte Vorgangsnummer \\
  Dateiname: & \texttt{fehlerfall\_12.in} \\
  Beschreibung: & Dieser Fehlerfall basiert auf der g\"ultigen Datei
  \texttt{ihk\_01.in}, nur dass in Zeile 12 die Vorgangsnummer 1
  doppelt vergeben wurde. \\
  Diskussion: & G\"ultige Vorgangsnummern sind nach Abschnitt
  \ref{vorgang} immer eindeutig. Das Programm erkennt hier also
  zurecht eine ung\"ultige Vorgangsnummer und beendet sich mit einem
  Fehler.\\
  Fehlermeldung: & \texttt{Fehler beim Einlesen in Zeile 12:
    Vorgangsnummer 1 mehrfach vorhanden!} \\
\end{tabularx}
\begin{tabularx}{\textwidth}{l X}
  Name: & Ung\"ultige Nachfolger \\
  Dateiname: & \texttt{fehlerfall\_13.in} \\
  Beschreibung: & Dieser Fehlerfall basiert auf der g\"ultigen Datei
  \texttt{ihk\_01.in}, nur dass in Zeile 8 ein ung\"ultiger
  Nachfolger, n\"amlich die Zeichenkette ``falsch'',
  eingetragen wurde. \\
  Diskussion: & G\"ultige Nachfolger sind nach Abschnitt
  \ref{vorgang} immer g\"ultige Vorgangsnummern, die durch ein Komma
  getrennt wurden. Das Programm erkennt hier also zurecht einen
  ung\"ultigen Nachfolger und beendet sich mit einem Fehler.\\
  Fehlermeldung: & \texttt{Fehler beim Einlesen in Zeile 8: Ungültiger
    Nachfolger falsch} \\
\end{tabularx}
\hrule
\begin{tabularx}{\textwidth}{l X}
  Name: & Nicht existierender Nachfolger \\
  Dateiname: & \texttt{fehlerfall\_14.in} \\
  Beschreibung: & Dieser Fehlerfall basiert auf der g\"ultigen Datei
  \texttt{ihk\_01.in}, nur dass in Zeile 7 bei Vorgang 2 der nicht
  existierende Nachfolger 8 eingetragen wurde. \\
  Diskussion: & Dieser Testfall zeigt, dass das Programm in der Lage
  ist, nicht existierende Nachfolger zu erkennen. \\
  Fehlermeldung: & \texttt{Fehler bei der Erstellung des Netzplans:
    Vorgang 2 hat Vorgang 8 als Nachfolger, obwohl dieser nicht
    existiert!} \\
\end{tabularx}
\hrule
\begin{tabularx}{\textwidth}{l X}
  Name: & Ung\"ultige Vorg\"angerbeziehung \\
  Dateiname: & \texttt{fehlerfall\_15.in} \\
  Beschreibung: & Dieser Fehlerfall basiert auf der g\"ultigen Datei
  \texttt{ihk\_01.in}, nur dass in Zeile 6 bei Vorgang 1 der Vorgang 2
  aus der Liste seiner Nachfolger entfernt wurde. \\
  Diskussion: & Dieser Testfall zeigt, dass das Programm eine
  Inkonsistenz erkennt, wenn
  ein Vorgang einen Vorg\"anger angibt, dessen Nachfolger er aber
  nicht ist. \\
  Fehlermeldung: & \texttt{Fehler bei der Erstellung des Netzplans:
    Inkonsistente Beziehung gefunden! Vorgang 2 hat Vorgang 1 als
    Vorgänger, ist aber selbst nicht Nachfolger von diesem!} \\
\end{tabularx}
\begin{tabularx}{\textwidth}{l X}
  Name: & Nicht zusammenh\"angender Plan \\
  Dateiname: & \texttt{fehlerfall\_16.in} \\
  Beschreibung: & Dieser Fehlerfall enth\"alt ein Projekt, welches aus
  nicht zusammenh\"angenden Einzelteilen besteht. Hier bilden die
  Vorg\"ange 1 und 2, sowie die Vorg\"ange 3 und 4 jeweils eine Kette.\\
  Diskussion: & Mit diesem Testfall wird \"uberpr\"uft, ob das
  Programm erkennt, wenn ein Netzplan nicht zusammenh\"angt. An dieser
  Stelle beendet sich das Programm richtigerweise mit der Meldung,
  dass man ausgehend von Vorgang 1 weder Vorgang 3 noch 4 erreichen
  kann, selbst wenn man entgegen der Pfeilspitzen l\"auft (was hier
  mit ungerichtet gemeint ist).\\
  Fehlermeldung: & \texttt{Fehler bei der Erstellung des Netzplans:
    Der Netzplan ist nicht zusammenhängend! Es existiert kein
    ungerichteter Pfad zwischen Vorgang 1 und den Vorgängen 3, 4} \\
\end{tabularx}
\hrule
\begin{tabularx}{\textwidth}{l X}
  Name: & Zwei einzelne Knoten \\
  Dateiname: & \texttt{fehlerfall\_17.in} \\
  Beschreibung: & Dieses Projekt besteht ebenfalls aus nicht
  zusammenh\"angenden Einzelteilen. Dieses Mal sind es nur zwei
  einzelne Knoten.\\
  Diskussion: & Dieser Testfall dient als eine Art Minimalbeispiel zu
  Fehlerfall Nr. 16. Hier wird auch der kleinste nicht
  zusammenh\"angende Netzplan erkannt.\\
  Fehlermeldung: & \texttt{Fehler bei der Erstellung des Netzplans:
    Der Netzplan ist nicht zusammenhängend! Es existiert kein
    ungerichteter Pfad zwischen Vorgang 1 und 2} \\
\end{tabularx}
\hrule
\begin{tabularx}{\textwidth}{l X}
  Name: & Kein Startknoten vorhanden \\
  Dateiname: & \texttt{fehlerfall\_18.in} \\
  Beschreibung: & In diesem Projekt wurde aus der g\"ultigen Datei
  \texttt{ihk\_01.in} der Startknoten 1 entfernt.\\
  Diskussion: & Mit diesem Testfall wird sichergestellt, dass das
  Programm auch erkennt, wenn ein Netzplan keinen Startknoten
  enth\"alt.\\
  Fehlermeldung: & \texttt{Fehler bei der Erstellung des Netzplans: Es
    existiert kein Startvorgang!} \\
\end{tabularx}
\begin{tabularx}{\textwidth}{l X}
  Name: & Kein Endknoten vorhanden \\
  Dateiname: & \texttt{fehlerfall\_19.in} \\
  Beschreibung: & In diesem Projekt wurde aus der g\"ultigen Datei
  \texttt{ihk\_01.in} der Endknoten 7 entfernt.\\
  Diskussion: & Mit diesem Testfall wird sichergestellt, dass das
  Programm auch erkennt, wenn ein Netzplan keinen Endknoten
  enth\"alt.\\
  Fehlermeldung: & \texttt{Fehler bei der Erstellung des Netzplans: Es
    existiert kein Endvorgang!} \\
\end{tabularx}
\hrule
\begin{tabularx}{\textwidth}{l X}
  Name: & Minimaler Zyklus \\
  Dateiname: & \texttt{fehlerfall\_20.in} \\
  Beschreibung: & In diesem Projekt existieren drei Vorg\"ange, wobei
  der mittlere Vorgang der Kette (Vorgang 2) einen Verweis auf sich
  selbst enth\"alt.\\
  Diskussion: & Dieser Testfall dient als Minimalbeispiel der
  Erkennung von Zyklen im Netzplan. Auch Knoten, die mit sich selbst
  Zyklen bilden, werden vom Programm erkannt.\\
  Fehlermeldung: & \texttt{Fehler bei der Erstellung des Netzplans: Es wurde ein Zyklus erkannt!
    2->2} \\
\end{tabularx}
\hrule
\begin{tabularx}{\textwidth}{l X}
  Name: & Doppelter Nachfolger \\
  Dateiname: & \texttt{fehlerfall\_21.in} \\
  Beschreibung: & In diesem Projekt, welches wieder auf der g\"ultigen
  Datei \texttt{ihk\_01.in} basiert, hat Vorgang 2 (Zeile 7) einen doppelten
  Nachfolger (Vorgang 3).\\
  Diskussion: & Doppelte Nachfolger deuten auf Fehler bei der Eingabe
  hin und sollten daher vom Programm erkannt und gemeldet werden.\\
  Fehlermeldung: & \texttt{Fehler beim Einlesen in Zeile 7: Doppelter
    Nachfolger 3} \\
\end{tabularx}
\hrule
\begin{tabularx}{\textwidth}{l X}
  Name: & Doppelter Vorg\"anger \\
  Dateiname: & \texttt{fehlerfall\_22.in} \\
  Beschreibung: & In diesem Projekt, welches wieder auf der g\"ultigen
  Datei \texttt{ihk\_01.in} basiert, hat Vorgang 2 (Zeile 7) einen doppelten
  Vorg\"anger (Vorgang 1).\\
  Diskussion: & Doppelte Vorg\"anger deuten auf Fehler bei der Eingabe
  hin und sollten daher vom Programm erkannt und gemeldet werden.\\
  Fehlermeldung: & \texttt{Fehler beim Einlesen in Zeile 7: Doppelter
    Vorgänger 1} \\
\end{tabularx}
\begin{tabularx}{\textwidth}{l X}
  Name: & Keine Vorg\"ange \\
  Dateiname: & \texttt{fehlerfall\_23.in} \\
  Beschreibung: & Diese Datei enth\"alt zwar Kommentarzeilen und eine
  g\"ultige \"Uberschriftzeile, allerdings keine Datenzeilen.\\
  Diskussion: & Jede g\"ultige Eingabedatei muss mindestens eine
  Datenzeile enthalten. Die Meldung des Fehlers erfolgt hier, indem
  darauf hingewiesen wird, dass kein Startvorgang existiert. Somit
  bedurfte dieser Testfall keiner weiteren \"Anderung im Programm, da
  er schon durch die Pr\"ufung auf Startvorg\"ange abgehandelt werden
  konnte.\\
  Fehlermeldung: & \texttt{Fehler bei der Erstellung des Netzplans: Es
    existiert kein Startvorgang!} \\
\end{tabularx}
\hrule
\begin{tabularx}{\textwidth}{l X}
  Name: & Die leere Datei \\
  Dateiname: & \texttt{fehlerfall\_24.in} \\
  Beschreibung: & Die leere Eingabedatei.\\
  Diskussion: & Hier wird sichergestellt, dass sich das Programm auch
  bei einer leeren Eingabedatei noch korrekt beendet. Dieser
  Fehlerfall wurde schon durch die Pr\"ufung der \"Uberschrift
  abgefangen, was die Fehlermeldung erkl\"art.\\
  Fehlermeldung: & \texttt{Fehler beim Einlesen: Keine Überschrift
    gefunden!} \\
\end{tabularx}
