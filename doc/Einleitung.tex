\chapter{Einleitung}
\label{Einleitung}

Die Kunst einer erfolgreichen zeitlichen Planung von Projekten ist es, die
einzelnen Schritte bis zum Ziel optimal aufeinander abzustimmen. Kommt
es in einer Kette von Aufgaben auch nur an einer Stelle zu
Verz\"ogerungen, kann dies das gesamte Projekt gef\"ahrden. Damit
Projektmanager auch in schwierigen Situationen stets den \"Uberblick
behalten k\"onnen, gibt es einige Werkzeuge, die bei der Planung gro{\ss}er
Projekte von besonderer Bedeutung sind. Eines dieser Werkzeuge soll
nun im Rahmen dieser Arbeit n\"aher untersucht werden: Der Netzplan.

\section{Was ist ein Netzplan?}

Ein Netzplan dient der Darstellung aller Abh\"angigkeiten zwischen den
einzelnen Vorg\"angen eines Projektes. Ein Vorgang innerhalb eines
Projektes l\"asst sich hierbei als einzelner Knoten in einer
verzweigten Kette aus weiteren Vorg\"angen auffassen, die alle
zusammen wiederum das gesamte Projekt repr\"asentieren. Die
Abh\"angigkeiten unter den Vorg\"angen werden hierbei durch Pfeile
ausgedr\"uckt: Ein Vorg\"anger besitzt einen Pfeil hin zu seinem
Nachfolger. Jeder Vorgang verf\"ugt hierbei \"uber bestimmte
Eigenschaften, die ihn eindeutig beschreiben:
\begin{itemize}
  \item \textbf{Vorgangsnummer:} Eine eindeutige Identifikationsnummer
    des Vorgangs
  \item \textbf{Vorgangsbezeichnung:} Eine lesbare Bezeichnung des Vorgangs
  \item \textbf{Dauer:} Wie lange dauert die Ausf\"uhrung des
    Vorgangs?
  \item \textbf{Vorg\"anger:} Welche anderen Vorg\"ange m\"ussen
    zuerst ausgef\"uhrt werden, bevor der aktuelle Vorgang
    ausgef\"uhrt werden kann?
  \item \textbf{Nachfolger:} Welche anderen Vorg\"ange warten auf eine
    Abarbeitung des aktuellen Vorgangs?
  \item \textbf{Fr\"uhester Anfangszeitpunkt (FAZ):} Wann kann
    fr\"uhestens mit der Bearbeitung des Vorgangs begonnen werden?
  \item \textbf{Sp\"atester Anfangszeitpunkt (SAZ):} Wann muss
    sp\"atestens mit der Bearbeitung begonnen werden, damit der
    Zeitplan nicht gef\"ahrdet wird?
  \item \textbf{Fr\"uhester Endzeitpunkt (FEZ):} Wann kann der Vorgang
    fr\"uhestens abgeschlossen werden?
  \item \textbf{Sp\"atester Endzeitpunkt (SEZ):} Wann muss der Vorgang
    sp\"atestens abgeschlossen sein, damit der Zeitplan nicht
    gef\"ahrdet wird?
  \item \textbf{Gesamtpuffer (GP):} Wie viel Spielraum liegt zwischen
    dem fr\"uhesten und dem sp\"atesten Anfangszeitpunkt?
  \item \textbf{Freier Puffer (FP):} Wie viel Spielraum existiert,
    wenn kein nachfolgender Vorgang durch den aktuellen Vorgang
    verz\"ogert werden soll?
\end{itemize}
Wichtig bei der Angabe der Vorg\"anger und Nachfolger ist, dass ein
Netzplan keine zyklischen Abh\"angigkeiten enthalten darf (sonst
k\"onnte sich die Dauer eines Projektes bis ins Unendliche hinziehen).
Zudem muss es stets mindestens einen Vorgang ohne Vorg\"anger (einen
\textit{Startvorgang}) und mindestens einen Vorgang ohne Nachfolger
(einen \textit{Endvorgang}) geben.

Da zu Beginn der Planungsphase im Normalfall noch nicht alle aufgelisteten
Attribute bekannt sind (i.d.R. nur Vorgangsnummer,
Vorgangsbezeichnung, Dauer, sowie Vorg\"anger und Nachfolger),
ist es die Aufgabe des Projektmanagers, die restlichen Eigenschaften
so zu bestimmen, dass der Zeitplan eingehalten
wird. An dieser Stelle kommt die Rolle des Netzplans zum tragen: Durch
die Verkettung der einzelnen Vorg\"ange mit ihren Vorg\"angern und
Nachfolgern k\"onnen die fehlenden Attribute (FAZ, SAZ, FEZ, SEZ, GP,
FP) systematisch berechnet werden. Hieraus ergibt sich auch
unmittelbar die Gesamtdauer des Projektes: Sie entspricht entweder dem
FEZ der Endvorg\"ange oder ist undefiniert, falls dieser nicht
eindeutig ist.

\section{Kritische Pfade}

In jedem Projekt gibt es immer auch Vorg\"ange, deren Abarbeitung an
besonders enge Fristen gekn\"upft ist und daher keinerlei Spielraum
erlaubt. Solche Vorg\"ange, bei denen entsprechend der obigen Notation
\(GP=0\) und \(FP=0\) gilt, werden im Folgenden als \textit{kritische
  Vorg\"ange} bezeichnet.
In jedem Projekt gibt es mindestens eine Kette von Vorg\"angen
beginnend bei einem Startvorgang hin zu einem Endvorgang, die
ausschlie{\ss}lich aus kritischen Vorg\"angen besteht. Eine derartige
Konstellation wird auch als \textit{kritischer Pfad} bezeichnet, da
eine reibungslose Abarbeitung dieser Vorg\"ange Grundvoraussetzung
f\"ur die Einhaltung des Zeitplans ist. Der Projektmanager sollte also
schon zu Beginn alle kritischen Pfade identifizieren und besonders auf
die termingerechte Abarbeitung der zugeh\"origen Vorg\"ange
achten. Auch bei dieser Aufgabe hilft ihm der Netzplan, der sich
systematisch nach kritischen Pfaden durchsuchen l\"asst.

\section{Ziel dieser Arbeit}

Das Ziel dieser Arbeit ist es nun, die Erstellung von Netzpl\"anen
sowie die Suche nach kritischen Pfaden zu automatisieren.
Das zu diesem Zweck zu entwickelnde Programm soll in der Lage sein, die
einzelnen Vorg\"ange eines Projektes aus einer Eingabedatei einzulesen
und hieraus einen korrekten Netzplan zu konstruieren. Die fehlenden
Eigenschaften der Vorg\"ange (FAZ, SAZ, FEZ, SEZ, GP, FP) sollen dazu
systematisch errechnet werden. Anschlie{\ss}end soll der Netzplan auf
kritische Pfade durchsucht werden, die Ergebnisse der Suche, sowie die
vollst\"andige Beschreibung aller Vorg\"ange und ihrer Fristen
inklusive der Gesamtdauer des Projektes, sollen
abschlie{\ss}end in einer Ausgabedatei gespeichert werden.
