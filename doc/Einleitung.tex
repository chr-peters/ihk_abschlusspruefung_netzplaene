\chapter{Einleitung}
\label{Einleitung}

Die Kunst einer erfolgreichen zeitlichen Planung von Projekten ist es, die
einzelnen Schritte bis zum Ziel optimal aufeinander abzustimmen. Kommt
es in einer Kette von Aufgaben auch nur an einer Stelle zu
Verz\"ogerungen, kann dies das gesamte Projekt gef\"ahrden. Damit
Projektmanager auch in schwierigen Situationen stets den \"Uberblick
behalten k\"onnen, gibt es einige Werkzeuge, die bei der Planung gro{\ss}er
Projekte von besonderer Bedeutung sind. Eines dieser Werkzeuge soll
nun im Rahmen dieser Arbeit n\"aher untersucht werden: Der Netzplan.
