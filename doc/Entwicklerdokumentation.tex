\chapter{Entwicklerdokumentation}
\label{Entwicklerdokumentation}

\section{Entwicklungsumgebung}
Die Entwicklung der Software wurde in folgender Umgebung vorgenommen:
\makeenvironment
Weiterhin wurde zum \"Ubersetzen der Software das
Build-Management-Tool \textit{Ant} in der \textit{Version 1.9.6}
verwendet. Das Skript \texttt{run\_testcases.ksh}, welches der
Hintereinanderausf\"uhrung aller Testf\"alle dient, wurde auf Basis
des \textit{Kornshell-Interpreters} verfasst.

\section{Aufbau der Software}

\subsection{Klassenstruktur}

Die Klassenstruktur der Software wird an dieser Stelle durch das in
Abbildung \ref{klassendiagramm} gezeigte UML-Klassendiagramm
beschrieben. N\"ahere Informationen zur Funktionsweise der einzelnen
Methoden k\"onnen bei Bedarf dem ausf\"uhrlich kommentierten Quelltext
entnommen werden.

\begin{figure}
  \resizebox{\textwidth}{!}{
    \tikzset{font=\ttfamily}
\renewcommand{\umltextcolor}{black}
\renewcommand{\umldrawcolor}{black}
\renewcommand{\umlfillcolor}{white}
\begin{tikzpicture}%[show background grid]
  \begin{class}[text width=13.5cm]{Vorgang}{0,0}
    \attribute{-nummer: int}
    \attribute{-bezeichnung: String}
    \attribute{-dauer: int}
    \attribute{-vorgaenger: int [0..*]}
    \attribute{-nachfolger: int [0..*]}
    \attribute{-faz: int}
    \attribute{-fez: int}
    \attribute{-saz: int}
    \attribute{-sez: int}
    \attribute{-gp: int}
    \attribute{-fp: int}

    \operation{+Vorgang(nummer: int, bezeichnung: String, dauer: int)}
    \operation{+addVorgaenger(vorgaenger: int)}
    \operation{+addNachfolger(nachfolger: int)}
    \operation{+istKritisch(): boolean}
    \operation{+toString(): String}
  \end{class}
  \begin{class}[text width=11cm]{Netzplan}{-1.25, -10}
    \attribute{-vorgaenge: Vorgang [1..*]}
    \attribute{-adjazenzen: int [n][n]}
    \attribute{-startKnoten: int[1..*]}
    \attribute{-endKnoten: int [1..*]}
    \attribute{-toInternal: Map\textless Integer, Integer\textgreater}
    \attribute{-fromInternal: Map\textless Integer, Integer\textgreater}

    \operation{+Netzplan(vorgaenge: Vorgang [1..*])}
    \operation{+getDauer(): int}
    \operation{+getKritischePfade(): List\textless List\textless Integer\textgreater\textgreater}
    \operation{-erzeugeAdjazenzen()}
    \operation{-istZyklenfrei(): boolean}
    \operation{-istZusammenhaengend(): boolean}
    \operation{-vorwaertsRechnung()}
    \operation{-rueckwaertsRechnung()}
    \operation{-zeitreserven()}
  \end{class}
  \begin{class}[text width=8cm]{VorgangLeser}{12, 0}
    \attribute{-vorgaenge: Vorgang [1..*]}
    \attribute{-ueberschrift: String}

    \operation{+VorgangLeser(datei: String)}
    \operation{+getVorgaenge(): Vorgang [1..*]}
    \operation{+getUeberschrift(): String}
    \operation{-leseVorgaenge(datei: String)}
  \end{class}
  \begin{class}[text width=13cm]{ProjektReport}{11.5, -10}
    \attribute{-datei: String}

    \operation{+ProjektReport(datei: String)}
    \operation{+erzeugeReport(plan: Netzplan, ueberschrift: String)}
  \end{class}

  \aggregation{VorgangLeser}{}{}{Vorgang}
  \aggregation{Netzplan}{}{}{Vorgang}

  %% \begin{interface}[text width=7.5cm]{FieldDescriptor}{-4, -4}
  %%   \operation[0]{getStrength( location: Vector3D ): Vector3D}
  %% \end{interface}
  
  %% \begin{interface}[text width=6.5cm]{Condition}{4, -4}
  %%   \operation[0]{check( track: vector<Particle> ): bool}
  %% \end{interface}

  %% \begin{class}[text width=6cm]{MaximumDistanceCondition}{-4, -7}
  %%   \implement{Condition}
  %%   \attribute{- maxDistance: double}
  %%   \attribute{- referencePoint: Vector3D}
  %% \end{class}

  %% \begin{class}[text width=6cm]{PlaneIntersectionCondition}{4, -7}
  %%   \implement{Condition}
  %%   \attribute{- plane: Plane3D}
  %% \end{class}

  %% \draw[umlcd  style  dashed  line ,->] (Propagator) --node[above , sloped ,
  %%   black]{} (FieldDescriptor);

  %% \draw[umlcd  style  dashed  line ,->] (Propagator) --node[above , sloped ,
  %%   black]{} (Condition);
\end{tikzpicture}

  }
  \caption{Der grundlegende Aufbau der verwendeten Klassen.}
  \label{klassendiagramm}
\end{figure}

\subsection{Softwarearchitektur}

Der architektonische Aufbau der Software entspricht in etwa den
untersten beiden Schichten einer
\textit{Drei-Schichten-Architektur}. Die
\textit{Pr\"asentationsschicht} wurde
hier bewusst nicht ber\"ucksichtigt, da w\"ahrend des eigentlichen
Programmablaufs keine Benutzerinteraktion stattfindet. Sie kann
allerdings aufbauend auf den existierenden Schichten
\textit{Datenhaltung} und \textit{Logik} problemlos aufgesetzt werden.
Die Klassen des Programms k\"onnen den einzelnen Schichten wie folgt
zugeordnet werden:
\begin{itemize}
  \item \textbf{Datenhaltungsschicht:} VorgangLeser und ProjektReport
    sind teile der Datenhaltungsschicht, da sie f\"ur die
    Kommunikation des Programms mit einem persistenten Datentr\"ager
    verantwortlich sind.
  \item \textbf{Logikschicht:} Hier sind die Klassen Vorgang und
    Netzplan anzusiedeln, da die tats\"achlichen Berechnungen
    innerhalb dieser Klassen (haupts\"achlich in der Klasse Netzplan)
    vollzogen werden.
\end{itemize}

\section{\"Ubersetzen der Software}

Um die Software zu \"ubersetzen, empfiehlt es sich, das
Build-Management-Tool \textit{Ant} in der genannten Version zu
verwenden. Gibt man den Befehlt \texttt{ant} im Verzeichnis der
\texttt{build.xml}-Datei in der Kommandozeile ein, so erzeugt Ant
automatisch zwei neue Verzeichnisse: Das Verzeichnis \texttt{build},
in welchem die \"ubersetzten Dateien abgelegt werden und das
Verzeichnis \texttt{dist}, in welchem das ausf\"uhrbare Programm in
Form einer \texttt{.jar}-Datei erzeugt wird.
Durch einen Aufruf von \texttt{ant clean} werden alle im Zuge der
\"Ubersetzung erzeugten Dateien und Verzeichnisse wieder entfernt.

\subsection{\"Ubersetzen ohne den Einsatz von Ant}

M\"ochte man das Programm ohne den Einsatz von Ant \"ubersetzen, so
sollte zun\"achst manuell ein Verzeichnis (z.B. \texttt{build})
erzeugt werden, in welchem die \"ubersetzten Dateien vom Compiler
abgelegt werden k\"onnen. Ausgehend vom Hauptverzeichnis kann nun
durch den Befehl
\begin{verbatim}
javac -d build/ -sourcepath src/ src/netzplanerstellung/Main.java
\end{verbatim}
das Programm \"ubersetzt werden.

\subsubsection{Erzeugen einer ausf\"uhrbaren .jar-Datei ohne Ant}

Um ohne den Einsatz von Ant eine ausf\"uhrbare .jar-Datei erzeugen zu
k\"onnen, muss zun\"achst ein Manifest erzeugt werden, welches in
jeder ausf\"uhrbaren .jar-Datei vorhanden sein muss und zumindest
Informationen \"uber die Hauptklasse enth\"alt. Es ist hierzu
ausreichend, die folgenden Informationen in einer Datei
\texttt{MANIFEST.MF} abzulegen:
\lstset{
  numbers=left
}
\begin{lstlisting}
  Main-Class: netzplanerstellung.Main
  <Leerzeile>
\end{lstlisting}
Man beachte, dass die \texttt{MANIFEST.MF} Datei mit einer Leerzeile
enden muss.
Soll die .jar-Datei nicht im aktuellen Verzeichnis erzeugt werden, so
kann zun\"achst ein Ordner \texttt{dist} angelegt werden. Mit dem
folgenden Befehl kann anschlie{\ss}end in diesem Ordner die
ausf\"uhrbare .jar-Datei erzeugt werden:
\begin{verbatim}
jar cmf MANIFEST.MF dist/netzplanerstellung-1.0.0.jar -C build .
\end{verbatim}

