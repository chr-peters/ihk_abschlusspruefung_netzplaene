\chapter{Verfahrensbeschreibung}
\label{Verfahrensbeschreibung}

\section{Datenstrukturen}

Die Grundlage eines jeden Programms ist durch die Art der verwendeten
Datenstrukturen gegeben. Damit im weiteren Verlauf Netzpl\"ane
effizient erzeugt werden k\"onnen, ist eine solide logische
Repr\"asentation unerl\"asslich. Die grundlegende Darstellung der
wichtigsten Elemente dieses Programms soll nun im Folgenden n\"aher
beschrieben werden.


\subsection{Logische Repr\"asentation eines Vorgangs}
\label{vorgang}
Ein einzelner Vorgang kann als Objekt dargestellt werden, welches die
in \ref{sec:netzplan} beschriebenen Attribute \textit{Vorgangsnummer,
  Vorgangsbezeichnung, Dauer, Vorg\"anger, Nachfolger, FAZ, SAZ, FEZ,
  SEZ, GP, FP}  enth\"alt und einen
Zugriff auf diese erm\"oglicht. Hierbei werden jedem Attribut die
folgenden Definitionsbereiche zugeordnet, welche sp\"ater vom Anwender
eingehalten werden m\"ussen:
\begin{itemize}
  \item \textbf{Vorgangsnummer:} Element der ganzen Zahlen \(\mathbb{Z}\),
    welches im gesamten Projekt \textit{eindeutig} sein muss.
  \item \textbf{Vorgangsbezeichnung:} Nicht leere Zeichenkette, welche
    das Semikolon nicht enthalten darf\footnote{Dies h\"angt mit dem
    Eingabeformat zusammen, was an sp\"aterer Stelle erl\"autert
    wird.}.
  \item \textbf{Dauer:} Element der nat\"urlichen Zahlen
    \(\mathbb{N}\). Dies bedeutet, dass keine negativen Dauern und
    auch keine nicht existenten Dauern zugelassen sind.
  \item \textbf{Vorg\"anger:} Liste g\"ultiger Vorgangsnummern.
  \item \textbf{Nachfolger:} Liste g\"ultiger Vorgangsnummern.
  \item \textbf{FAZ, SAZ, FEZ, SEZ, GP und FP:} Elemente der nat\"urlichen
    Zahlen \(\mathbb{N}\) \textit{inklusive} der \(0\).
\end{itemize}
Mehrere Vorg\"ange, welche zusammen ein Projekt bilden, k\"onnen somit in
einer Liste aus einzelnen Vorgangsobjekten abgelegt werden.


\subsection{Logische Repr\"asentation eines Netzplans}

Ein Netzplan kann formal als gerichteter azyklischer Graph beschrieben
werden, wobei die Knotenmenge durch die Menge aller Vorg\"ange des
Projektes gegeben ist und die Kantenmenge durch die
Vorg\"anger-Nachfolger-Relation unter den Vorg\"angen beschrieben
werden kann.
Ein Hilfsmittel, welches zus\"atzlich zu der Liste aller Vorg\"ange
des Projekts
verwendet wird, um den Netzplan als Graph darzustellen, ist die
sogenannte \textit{Adjazenzmatrix}. Diese Matrix \(A\) besitzt f\"ur jeden
Knoten des Graphen sowohl eine Zeile als auch eine Spalte und hat
daher bei \(n\) Knoten die Dimension \(n \times n\). Existiert eine direkte
Verbindung von Knoten \(i\) nach Knoten \(j\) (dies ist anschaulich
der Fall, wenn Vorgang \(j\) Nachfolger von Vorgang \(i\) ist),
so ist der jeweilige Eintrag \(a_{ij}\) der Adjazenzmatrix \(1\),
ansonsten \(0\). Anhand der Adjazenzmatrix l\"asst sich auch leicht
erkennen, ob ein Knoten Start- oder Endknoten ist: Da ein Startknoten \(i\)
keinen Vorg\"anger hat, gilt f\"ur die entsprechenden Eintr\"age der
\(i\)'ten Spalte: \(a_{ki} = 0 \forall 1 \leq k \leq n\).
F\"ur jeden Endknoten \(j\) gilt analog \(a_{jk} = 0 \forall 1 \leq k
\leq n\).

Es sei an dieser Stelle schon angemerkt, dass Vorg\"ange in ihrer
Nummerierung nicht zwingend die Zahlen \(1\) bis \(n\) (oder sp\"ater
im Rechner \(0\) bis \(n-1\)) belegen m\"ussen. Daher wird
zus\"atzlich eine Abbildung \(g: Vorgangsnummern \rightarrow
\mathbb{N}\) eingef\"uhrt, welche die Vorgangsnummern
l\"uckenlos auf die ersten \(n\) nat\"urlichen Zahlen
abbildet. Dementsprechend bezeichnet \(g^{-1}\) die zugeh\"orige
Umkehrabbildung, welche eine Zahl von \(1\) bis \(n\) auf die
entsprechende urspr\"ungliche Vorgangsnummer abbildet.

\section{Einlesen der Vorg\"ange}

\subsection{Format der Eingabedatei}

Das Format, in dem die Daten dem Programm zugef\"uhrt werden, soll hier
anhand des folgenden Beispiels erl\"autert werden:

\begin{figure}[h!]
  \resizebox{\textwidth}{!}{
    \fbox{
      \setlength{\fboxrule}{1pt}
      \lstinputlisting{../testcases/ihk_01.in}
    }
  }
  \caption{Beispiel einer Eingabedatei}
\end{figure}
Eine Datei besteht aus einer ungeordneten Menge an Kommentar- und
Datenzeilen. Leerzeilen sind \textit{nicht}
zugelassen. Kommentarzeilen lassen den Anwender Anmerkungen
hinzuf\"ugen, welche vom Programm nicht beachtet werden. Sie beginnen
stets mit einem \texttt{//}. Eine besondere Art der Kommentarzeile ist
die \textit{\"Uberschriftzeile}, da sie im Gegensatz zu den normalen
Kommentarzeilen vom Programm besonders ber\"ucksichtigt wird. Sie
beginnt immer mit einem \texttt{//+} und kann an beliebiger Stelle
innerhalb der Datei auftauchen. Die Zeichenkette, welche auf das
einleitende \texttt{//+} folgt, beschreibt das jeweilige Projekt und
darf nicht leer sein. Pro Datei muss es \textit{genau eine}
\"Uberschriftzeile geben. Im gegebenen Beispiel sind die Zeilen 1, 3
und 4 klassische Kommentarzeilen und werden vom Programm nicht
beachtet. Zeile 2 ist die \"Uberschriftzeile.

Datenzeilen liefern dem Programm die notwendigen Informationen zu den
einzelnen Vorg\"angen eines Projekts. Jede Datenzeile beschreibt genau
einen Vorgang und besteht aus f\"unf Elementen, welche jeweils durch
ein Semikolon getrennt werden: \textit{Vorgangsnummer,
  Vorgangsbezeichnung, Dauer, Vorg\"anger und
  Nachfolger}\footnote{Dies ist auch der Grund, weshalb im
  Definitionsbereich der Vorgangsbezeichnung keine Semikola zugelassen
  sind.}.
Die Eingaben zu Vorgangsnummer, Vorgangsbezeichnung und Dauer m\"ussen
hierbei mit den Definitionsbereichen der in \ref{vorgang} beschriebenen Attribute
eines Vorgangs \"ubereinstimmen. Die Elemente Vorg\"anger und
Nachfolger werden jeweils durch g\"ultige Vorgangsnummern angegeben,
welche durch Kommata getrennt aufgez\"ahlt werden. Doppelte
Vorg\"anger oder Nachfolger sind nicht erlaubt. Hat ein Vorgang
keinen Vorg\"anger bzw. keinen Nachfolger, so steht an dieser Stelle
das Zeichen \texttt{-}. Es handelt sich in diesem Fall um einen Start-
oder Endvorgang. Jedes Projekt muss mindestens einen Start- und einen
Endvorgang enthalten.

\subsection{Algorithmus zum Einlesen}

Der Algorithmus, der zum Einlesen der Eingabedatei implementiert
wurde, iteriert im Wesentlichen \"uber alle Zeilen und pr\"uft, um was
f\"ur eine Art der Zeile es sich handelt. Je nach Art der Zeile werden
die Eingaben validiert, stimmt etwas nicht, so wird ein Fehler
geworfen. W\"ahrend der Iteration wird sukzessive eine Liste mit
gelesenen Vorg\"angen bef\"ullt, die am Ende zur\"uckgegeben
wird. Diese Liste wird im folgenden Struktogramm (siehe Abbildung \ref{einlesen}), welches einen
\"Uberblick \"uber die wesentlichen Gesichtspunkte des Algorithmus
gibt, als \textit{Resultat} bezeichnet.
\begin{figure}[h]
  %\sProofOn
\resizebox{\textwidth}{!}{
\begin{struktogramm}(150, 122)
  \while{Solange noch Zeilen zu lesen sind}
  \ifthenelse{6}{3}{Zeile beginnt mit \texttt{//+}}{\sTrue}{\sFalse}
  \ifthenelse{3}{3}{\"Uberschrift in Ordnung?}{\sTrue}{\sFalse}
  \assign{Setze \"Uberschrift}
  \exit{Neue Zeile}
  \change
  \assign{Fehlermeldung: \"Uberschrift nicht in Ordnung}
  \ifend
  \change
  \ifend
  \ifthenelse{5}{5}{Zeile beginnt mit \texttt{//}}{\sTrue}{\sFalse}
    \exit{Neue Zeile}
    \change
  \ifend
  \assign{Trenne Zeile am Semikolon auf}
  \assign{Validiere die einzelnen Teile}
  \ifthenelse{5}{5}{Validierung erfolgreich?}{\sTrue}{\sFalse}
  \assign{Erzeuge hieraus ein Objekt vom Typ Vorgang}
  \assign{F\"uge den Vorgang zum Resultat hinzu}
  \change
  \assign{Fehlermeldung: Teile nicht in Ordnung}
  \ifend
  \whileend
  \return[8]{Gebe Resultat zur\"uck}
\end{struktogramm}
}
%\sProofOff

  \caption{Algorithmus zum Einlesen der Eingabedatei}
  \label{einlesen}
\end{figure}

\section{Konstruktion des Netzplans}

Um aus der eingelesenen Liste aus Vorgangsobjekten nun den Netzplan
konstruieren zu k\"onnen, sind einige Schritte notwendig, die an
dieser Stelle n\"aher beschrieben werden.

\subsection{Initialisierung}

W\"ahrend der Initialisierungsphase wird zun\"achst gepr\"uft, ob alle
Grundvoraussetzungen erf\"ullt sind, damit es sich um einen g\"ultigen
Netzplan handelt. Hierzu wird zun\"achst die Adjazenzmatrix erzeugt
und w\"ahrenddessen die Konsistenz der Vorg\"anger - Nachfolger -
Relation sichergestellt.

\subsubsection{Erzeugung der Adjazenzmatrix}

Um die Adjazenzmatrix zu erzeugen, wird jede Vorgangsnummer einer
Spalte in der Matrix zugeordnet. Hierzu wird die Abbildung \(g\)
verwendet, die bereits bei der Beschreibung des Netzplans erl\"autert
wurde. Man erh\"alt \(g\), indem man ein assoziatives Feld anlegt,
wobei jeder Vorgangsnummer die zugeh\"orige Position in der Liste aus
Vorg\"angen zugeordnet wird.
F\"ur jeden Vorgang wird nun in der zugeh\"origen Zeile der
Adjazenzmatrix in der Spalte
jedem seiner Nachfolger eine 1 in der Matrix eingetragen. Hierbei wird
auch gleich abgepr\"uft, ob die Beziehung eines Vorgangs zu seinen
Vorg\"angern und Nachfolgern konsistent ist: Ist z.B. ein Vorgang \(a\) als
Vorg\"anger von Vorgang \(b\) eingetragen, hat diesen aber nicht als
Nachfolger, so ist die Beziehung inkonsistent und der Netzplan kann
nicht erzeugt werden. Das Programm wird in diesem Fall mit einem
Fehler beendet.

\subsubsection{H\"angt der Graph zusammen?}

Nachdem die Adjazenzmatrix erzeugt wurde und die Konsistenz der
Beziehungen unter den Vorg\"angen sichergestellt ist, muss im
n\"achsten Schritt gepr\"uft werden, ob der Graph auch
zusammenh\"angt, da es sich nur in diesem Fall um einen g\"ultigen
Netzplan handelt. Um dies zu bewerkstelligen, wird der bis dahin
gerichtete Graph vor\"ubergehend als ungerichtet aufgefasst. Beginnend
bei einem beliebigen Knoten wird der Graph im n\"achsten Schritt
traversiert. Bleiben nach dem Durchlaufen noch Knoten \"ubrig, welche
nicht besucht wurden, war der Graph nicht zusammenh\"angend.

Um aus dem gerichteten Graph einen ungerichteten Graph zu erzeugen,
kann einfach eine symmetrische Version der Adjazenzmatrix erstellt
werden. Hierzu wird zu jedem Element \(a_{ij}=1\) das an der
Diagonalen gespiegelte Element \(a_{ji}\) ebenfalls auf 1 gesetzt.
Die Traversierung, welche im n\"achsten Schritt erfolgt, wird mittels
einer Breitensuche durchgef\"uhrt.

\subsubsection{\"Uberpr\"ufen auf Zyklen}

Da ein g\"ultiger Netzplan keine Zyklen enthalten darf, muss nun
gepr\"uft werden, ob der vorliegende Graph auch azyklisch ist. Hierzu
wird ausgehend von jedem Startknoten eine Expansion des Graphen
gebildet (d.h. es werden alle m\"oglichen Pfade durch den Graphen erzeugt).
Sobald ein Pfad im n\"achsten Schritt der Expansion ein Element
aufnehmen w\"urde, welches bereits in diesem Pfad enthalten ist, wurde
ein Zyklus entdeckt und das Programm kann mit einem Fehler beendet
werden. Auch hier wird die zu diesem Zweck ben\"otigte
Traversierung mit einer Breitensuche durchgef\"uhrt.

\subsection{Vorw\"artsrechnung}

Nachdem in den letzten Abschnitten sichergestellt wurde, dass aus der
Liste an Vorg\"angen ein g\"ultiger Netzplan erzeugt werden kann,
m\"ussen nun die noch fehlenden Eintr\"age FAZ, SAZ, FEZ, SEZ, GP und
FP f\"ur jeden Vorgang berechnet werden. In der Phase der
Vorw\"artsrechnung werden f\"ur jeden Vorgang die Werte FAZ und FEZ
gesetzt, weiterhin kann nach dieser Phase auch die Dauer des Projektes
abgelesen werden. Diese entspricht n\"amlich wie in Abschnitt
\ref{sec:netzplan} beschrieben dem FEZ der Endvorg\"ange, insofern
dieser eindeutig ist.

Um die Vorw\"artsrechnung durchzuf\"uhren, wird eine Warteschlange
\textit{(Queue)} mit
abzuarbeitenden Knoten erzeugt, welche zun\"achst nur die Startknoten
enth\"alt. Solange diese Queue noch Elemente hat, wird in jedem
Schritt das erste Element bearbeitet. Hierbei wird zuerst der FEZ des
Elements auf
FAZ+Dauer gesetzt. Man beachte hierbei, dass Startknoten von Beginn an
einen FAZ von 0 haben. Anschlie{\ss}end werden alle Nachfolger
des Elements betrachtet. Der FAZ eines jeden Nachfolgers wird nun auf
den FEZ des aktuellen Elements gesetzt, wenn sich der FAZ des
Nachfolgers hierdurch vergr\"o{\ss}ert. Jeder Nachfolger wird dann hinten
an die Queue der zu bearbeitenden Knoten angef\"ugt und eine neue
Iteration beginnt.

\subsection{R\"uckw\"artsrechnung}

In der Phase der R\"uckw\"artsrechnung werden nun f\"ur jeden Knoten
die Werte SAZ und SEZ gesetzt. Hierbei wird ebenfalls
eine Queue mit abzuarbeitenden Knoten erzeugt, welche diese Mal nur
die Endknoten enth\"alt. Bei jedem Endknoten wird nun SEZ=FEZ gesetzt,
da Endknoten keinen Puffer haben. Die Queue wird nun wieder beginnend
mit dem ersten Element abgearbeitet, bis sie leer ist. Hierbei wird
zun\"achst der SAZ des aktuellen Elementes auf SEZ-Dauer gesetzt. Im
Anschluss werden alle Vorg\"anger des Elementes betrachtet. Wurde der
SEZ des Vorg\"angers noch nicht gesetzt oder w\"urde er sich
verringern, so wird dieser auf den SAZ des aktuellen Elements
gesetzt. Anschlie{\ss}end wird wieder jeder Vorg\"anger hinten an die
Queue angeh\"angt und die Iteration beginnt von neuem.

\subsection{Berechnung der Zeitreserven}

Die Zeitreserven eines jeden Vorganges werden durch die Attribute GP
und FP ausgedr\"uckt. Um diese zu berechnen, werden alle Vorg\"ange
durchlaufen. Hierbei wird f\"ur jeden Vorgang GP=SAZ-FAZ=SEZ-FEZ
gesetzt. F\"ur den Wert FP gilt folgender Zusammenhang:
FP=\(\min\limits_{i \in \{Nachfolger\}} FAZ_i - FEZ\). Dieser wird
ebenfalls f\"ur jeden Vorgang ermittelt und gesetzt.

\section{Auffinden der kritischen Pfade}

Nachdem der Netzplan nun vollst\"andig erzeugt wurde, stellt sich noch
die Frage, wie die kritischen Pfade gefunden werden k\"onnen. Hierzu
wird mittels Tiefensuche (Breitensuche ist ebenfalls m\"oglich, im
Programmsystem \"andert sich hier nur eine einzige Zeile) ausgehend
von jedem kritischen Startknoten
jeder kritische Pfad erzeugt. Sobald ein Endknoten zu einem bis dahin
kritischen Pfad hinzugef\"ugt wird, ist ein kritischer Pfad
komplett und kann dem Resultat angef\"ugt werden. Ein \"Uberblick
dieses Algorithmus findet sich in Form eines Struktogramms in
Abbildung \ref{pfadefinden}.

\begin{figure}[h]
  %\sProofOn
\resizebox{\textwidth}{!}{
\begin{struktogramm}(150, 125)
  \assign{Erzeuge leere Liste \textit{Resultat} f\"ur die kritischen Pfade}
  \while{F\"ur jeden Startknoten \textit{aktStartKnoten}}
  \assign{Erzeuge leere Queue \textit{Pfade}, die abzuarbeitende Pfade enth\"alt}
  \ifthenelse{3}{3}{Ist \textit{aktStartKnoten} kritisch?}{\sTrue}{\sFalse}
  \assign{F\"uge ihn als ``Startpfad'' der Queue \textit{Pfade} hinzu}
  \change
  \exit{Betrachte n\"achsten Startknoten}
  \ifend
  \while{Solange \textit{Pfade} noch Elemente enth\"alt}
  \assign{Entferne erstes Element \textit{aktPfad}}
  \ifthenelse{3}{3}{Ist der letzte Knoten von \textit{aktPfad} ein Endknoten?}{\sTrue}{\sFalse}
  \assign{F\"uge \textit{aktPfad} der Liste \textit{Resultat} hinzu}
  \exit{Continue}
  \change
  \ifend
  \while{F\"ur jeden \textit{kritischen Nachfolger} des letzten Elements aus \textit{aktPfad}}
  \assign{Erzeuge \textit{neuerPfad} = \textit{aktPfad} + \textit{kritischer Nachfolger}}
  \assign{F\"uge \textit{neuerPfad} vorne an \textit{Pfade} an (\(\rightarrow\)Tiefensuche)}
  \whileend
  \whileend
  \whileend
  \assign{return \textit{Resultat}}
\end{struktogramm}
}
%\sProofOff

  \caption{Algorithmus zum Finden aller kritischen Pfade. Man beachte,
  dass der Unterschied zur Breitensuche nur in einer Zeile liegt: Bei
  der Breitensuche w\"urde \textit{neuerPfad} hinten an \textit{Pfade}
  angef\"ugt.}
  \label{pfadefinden}
\end{figure}

\section{Ausgabe der Ergebnisse}

Das Format, in dem die Ergebnisse vom Programm ausgegeben werden, soll
hier anhand des folgenden Beispiels erl\"autert werden, welches
gleichzeitig die Ausgabe zu der zuvor vorgestellten Eingabedatei bildet:

\begin{figure}[h!]
  \resizebox{\textwidth}{!}{
    \fbox{
      \setlength{\fboxrule}{1pt}
      \lstinputlisting{../testcases/ihk_01.out}
    }
  }
  \caption{Die resultierende Ausgabedatei}
\end{figure}

Das Format der Ausgabedatei wird durch die \"Uberschriftzeile
eingeleitet, welche nur den Text der anfangs eingelesenen
\"Uberschrift enth\"alt. Es folgt eine Leerzeile, woraufhin die
Spalten\"uberschriften des Ergebnisses ausgegeben
werden. Anschlie{\ss}end wird jeder Vorgang mit all seinen
Attributen ausgegeben (Vorg\"anger und Nachfolger ausgenommen).
Nach einer weiteren Leerzeile schlie{\ss}en sich die
Anfangsvorg\"ange, die Endvorg\"ange und die Gesamtdauer des Projekts
an. Es folgt wiederum eine
Leerzeile, woraufhin unter der entsprechenden \"Uberschrift die
kritischen Pfade untereinander ausgegeben werden.
