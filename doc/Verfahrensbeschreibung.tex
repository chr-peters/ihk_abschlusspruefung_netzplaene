\chapter{Verfahrensbeschreibung}
\label{Verfahrensbeschreibung}

\section{Datenstrukturen}

Die Grundlage eines jeden Programms ist durch die Art der verwendeten
Datenstrukturen gegeben. Damit im weiteren Verlauf Netzpl\"ane
effizient erzeugt werden k\"onnen, ist eine solide logische
Repr\"asentation unerl\"asslich. Die grundlegende Darstellung der
wichtigsten Elemente dieses Programms soll nun im Folgenden n\"aher
beschrieben werden.


\subsection{Logische Repr\"asentation eines Vorgangs}
\label{vorgang}
Ein einzelner Vorgang kann als Objekt dargestellt werden, welches die
in \ref{sec:netzplan} beschriebenen Attribute \textit{Vorgangsnummer,
  Vorgangsbezeichnung, Dauer, Vorg\"anger, Nachfolger, FAZ, SAZ, FEZ,
  SEZ, GP, FP}  enth\"alt und einen
Zugriff auf diese erm\"oglicht. Hierbei werden jedem Attribut die
folgenden Definitionsbereiche zugeordnet, welche sp\"ater vom Anwender
eingehalten werden m\"ussen:
\begin{itemize}
  \item \textbf{Vorgangsnummer:} Element der ganzen Zahlen \(\mathbb{Z}\),
    welches im gesamten Projekt \textit{eindeutig} sein muss.
  \item \textbf{Vorgangsbezeichnung:} Nicht leere Zeichenkette, welche
    das Semikolon nicht enthalten darf (dies h\"angt mit dem
    Eingabeformat zusammen, was an sp\"aterer Stelle erl\"autert
    wird).
  \item \textbf{Dauer:} Element der nat\"urlichen Zahlen
    \(\mathbb{N}\). Dies bedeutet, dass keine negativen Dauern und
    auch keine nicht existenten Dauern zugelassen sind.
  \item \textbf{Vorg\"anger:} Liste g\"ultiger Vorgangsnummern.
  \item \textbf{Nachfolger:} Liste g\"ultiger Vorgangsnummern.
  \item \textbf{FAZ, SAZ, FEZ, SEZ, GP und FP:} Elemente der nat\"urlichen
    Zahlen \(\mathbb{N}\) \textit{inklusive} der \(0\).
\end{itemize}
Mehrere Vorg\"ange, welche zusammen ein Projekt bilden, k\"onnen somit in
einer Liste aus einzelnen Vorgangsobjekten abgelegt werden.


\subsection{Logische Repr\"asentation eines Netzplans}

Ein Netzplan kann formal als gerichteter azyklischer Graph beschrieben
werden, wobei die Knotenmenge durch die Menge aller Vorg\"ange des
Projektes gegeben ist und die Kantenmenge durch die
Vorg\"anger-Nachfolger-Relation unter den Vorg\"angen beschrieben
werden kann.
Ein Hilfsmittel, welches zus\"atzlich zu der Liste aller Vorg\"ange
des Projekts
verwendet wird, um den Netzplan als Graph darzustellen, ist die
sogenannte \textit{Adjazenzmatrix}. Diese Matrix \(A\) besitzt f\"ur jeden
Knoten des Graphen sowohl eine Zeile als auch eine Spalte und hat
daher bei \(n\) Knoten die Dimension \(n \times n\). Existiert eine direkte
Verbindung von Knoten \(i\) nach Knoten \(j\) (dies ist anschaulich
der Fall, wenn Vorgang \(j\) Nachfolger von Vorgang \(i\) ist),
so ist der jeweilige Eintrag \(a_{ij}\) der Adjazenzmatrix \(1\),
ansonsten \(0\). Anhand der Adjazenzmatrix l\"asst sich auch leicht
erkennen, ob ein Knoten Start- oder Endknoten ist: Da ein Startknoten \(i\)
keinen Vorg\"anger hat, gilt f\"ur die entsprechenden Eintr\"age der
\(i\)'ten Spalte: \(a_{ki} = 0 \forall 1 \leq k \leq n\).
F\"ur jeden Endknoten \(j\) gilt analog \(a_{jk} = 0 \forall 1 \leq k
\leq n\).

Es sei an dieser Stelle schon angemerkt, dass Vorg\"ange in ihrer
Nummerierung nicht zwingend die Zahlen \(1\) bis \(n\) (oder sp\"ater
im Rechner \(0\) bis \(n-1\)) belegen m\"ussen. Daher wird
zus\"atzlich eine Abbildung \(g: Vorgangsnummern \rightarrow
\mathbb{N}\) eingef\"uhrt, welche die Vorgangsnummern
l\"uckenlos auf die ersten \(n\) nat\"urlichen Zahlen
abbildet. Dementsprechend bezeichnet \(g^{-1}\) die zugeh\"orige
Umkehrabbildung, welche eine Zahl von \(1\) bis \(n\) auf die
entsprechende urspr\"ungliche Vorgangsnummer abbildet.

\section{Einlesen der Vorg\"ange}

\subsection{Format der Eingabedatei}

Das Format, in dem die Daten dem Programm zugef\"uhrt werden, soll hier
anhand des folgenden Beispiels erl\"autert werden:

\begin{figure}[h!]
  \fbox{
    \setlength{\fboxrule}{1pt}
    \lstinputlisting{../testcases/ihk_01.in}
  }
  \caption{Beispiel einer Eingabedatei}
\end{figure}
Eine Datei besteht aus einer uneordneten Menge an Kommentar- und
Datenzeilen. Leerzeilen sind \textit{nicht}
zugelassen. Kommentarzeilen lassen den Anwender Anmerkungen
hinzuf\"ugen, welche vom Programm nicht beachtet werden. Sie beginnen
stets mit einem \texttt{//}. Eine besondere Art der Kommentarzeile ist
die \textit{\"Uberschriftzeile}, da sie im Gegensatz zu den normalen
Kommentarzeilen vom Programm besonders ber\"ucksichtigt wird. Sie
beginnt immer mit einem \texttt{//+} und kann an beliebiger Stelle
innerhalb der Datei auftauchen. Pro Datei muss es \textit{genau eine}
\"Uberschriftzeile geben. Im gegebenen Beispiel sind die Zeilen 1, 3
und 4 klassische Kommentarzeilen und werden vom Programm nicht
beachtet. Zeile 2 ist die \"Uberschriftzeile.

Datenzeilen liefern dem Programm die notwendigen Informationen zu den
einzelnen Vorg\"angen eines Projekts. Jede Datenzeile beschreibt genau
einen Vorgang und besteht aus f\"unf Elementen, welche jeweils durch
ein Semikolon getrennt werden: \textit{Vorgangsnummer,
  Vorgangsbezeichnung, Dauer, Vorg\"anger und
  Nachfolger}\footnote{Dies ist auch der Grund, weshalb im
  Definitionsbereich der Vorgangsbezeichnung keine Semikola zugelassen
  sind.}.
Die Eingaben zu Vorgangsnummer, Vorgangsbezeichnung und Dauer m\"ussen
hierbei mit den Definitionsbereichen der in \ref{vorgang} beschriebenen Attributen
eines Vorgangs \"ubereinstimmen. Die Elemente Vorg\"anger und
Nachfolger werden jeweils durch g\"ultige Vorgangsnummern angegeben,
welche durch Kommata getrennt aufgez\"ahlt werden. Hat ein Vorgang
keinen Vorg\"anger bzw. keinen Nachfolger, so steht an dieser Stelle
das Zeichen \texttt{-}. Es handelt sich in diesem Fall um einen Start-
oder Endvorgang.

\subsection{Algorithmus zum Einlesen}

\section{Konstruktion des Netzplans}

\section{Auffinden der kritischen Pfade}

\section{Ausgabe der Ergebnisse}
