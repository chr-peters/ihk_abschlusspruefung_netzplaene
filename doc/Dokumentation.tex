\documentclass{gropro}
\usepackage[utf8]{inputenc}

%\usepackage[ngerman]{babel}
\usepackage{amsfonts}
\usepackage{listings}
\usepackage[ngerman]{struktex}

% Umlaute fuer listings
\lstset{literate=%
  {Ö}{{\"O}}1
  {Ä}{{\"A}}1
  {Ü}{{\"U}}1
  {ß}{{\ss}}1
  {ü}{{\"u}}1
  {ä}{{\"a}}1
  {ö}{{\"o}}1
}
\lstset{basicstyle=\ttfamily}

\Autor{Christian Peters}
\PruefungsNR{20613}
\Ausbildungsort{Institut f\"ur Kernphysik\\Forschungszentrum J\"ulich}
\Thema{Automatisierte Erstellung von Netzpl\"anen}
\Programmiersprache{Java}
\Compiler{javac 1.8.0\_161}
\Rechner{Intel Core i7-4790 CPU @ 3.60GHz, 16GB RAM}
\Betriebssystem{Ubuntu 16.04 LTS}
%Mit includeonly koennen einzelne Kapitel seperat compiliert werden.
%Beispiel: \includeonly{Aufgabenanalyse} 
%         -> Das Dokument besteht nur aus dem Kapitel "Aufgabenanalyse" 
%\includeonly{}

\begin{document}

  \maketitle
  \pagenumbering{Roman} 
  \tableofcontents
  \cleardoublepage
  
  \pagenumbering{arabic}
  %\Ginclude{Aufgabenanalyse}
  \Ginclude{Einleitung}
  \Ginclude{Verfahrensbeschreibung}
  \Ginclude{Entwicklerdokumentation}
  \Ginclude{Benutzeranleitung}    
  %\Ginclude{Programmbeschreibung}
  \Ginclude{Testdokumentation}
  \Ginclude{Ausblick}
  \begin{appendix}
    \Ginclude{Abweichungen}
    %\makedeposition
    %\Ginclude{Aufgabenstellung_Deckblatt}
    %\Ginclude{Quellcode_Deckblatt}
    %\Ginclude{Test_Deckblatt}
  \end{appendix}

\end{document}
