\chapter{Ausblick}
\label{Ausblick}

An dieser Stelle sei abschlie{\ss}end noch eine Auswahl an Ideen
vermerkt, mit denen der
Funktionsumfang dieser Software in Zukunft noch erweitert, verbessert,
oder erg\"anzt werden k\"onnte.

\section{Implementierung der Pr\"asentationsschicht}

Da in der Planungsphase des Projekts keinerlei Benutzerinteraktion
angedacht war, wurde die Pr\"asentationsschicht aus der
Drei-Schichten-Architektur bisher vernachl\"assigt. Es empfiehlt sich
allerdings, diesen Sachverhalt zu \"andern, um die Software
benutzungsfreundlicher zu gestalten. Da keine der unteren Schichten in
der Drei-Schichten-Architektur von der Pr\"asentationsschicht
abh\"angt, lie{\ss}e sich diese als reine Erweiterung
auf der bisherigen Implementierung aufbauen. So kann es dem Benutzer
erleichtert werden, Dateien zur Bearbeitung auszuw\"ahlen und sich die
berechneten Ergebnisse auch graphisch darstellen zu lassen.

\section{Optimierung der Speicherung des Graphen}

Die gew\"ahlte Speicherung des Graphen als Adjazenzmatrix kann noch
optimiert werden. Der Vorteil einer Adjazenzmatrix ist sicherlich die
leichte Implementierung und das schnelle \"Uberpr\"ufen einer
Verbindung in \(\mathcal{O}(1)\). Auch das \"Uberpr\"ufen auf
Zusammenhang des Graphen lie{\ss} sich mithilfe der Adjazenzmatrix
leicht durchf\"uhren. Leider w\"achst der Speicherbedarf des Graphen
jedoch quadratisch mit der Anzahl der Knoten, wenn man eine
Adjazenzmatrix verwendet. Dies ist zwar in heutiger Zeit selbst bei
gro{\ss}en Projekten nicht das Problem (selbst bei 10000 Knoten
w\"urde die Matrix bei 4-Byte Integern nur ca. 400MB verbrauchen),
kann jedoch verbessert werden. Hierzu k\"onnte die Matrix
beispielsweise als d\"unn besetzte Matrix abgespeichert
werden, da die meisten Knoten untereinander nicht verbunden sind und
so viele Nulleintr\"age entstehen. Mithilfe des sogenannten \textit{Compressed Row
  Storage (CRS)}-Formats lie{\ss}e sich so der Speicheraufwand signifikant
reduzieren.

\section{Unterst\"utzung weiterer Ein- und Ausgabeformate}

Um die Interoperabilit\"at dieser Software mit anderen Systemen der
Projektplanung zu erleichtern, sollte das Repertoire an
unterst\"utzten Formaten erweitert werden. Es existieren aktuell viele
deklarative Dateiformate, welche sich gro{\ss}er Beliebtheit
erfreuen. Als Beispiele seien an dieser Stelle Formate wie XML oder
JSON genannt, die sich hervorragend dazu eignen, logische Strukturen
wie Vorg\"ange, Projekte und Netzpl\"ane darzustellen. Um diese neuen
Formate nun ohne gro{\ss}en Aufwand in das bestehende Programm
einzugliedern, sind ein paar Handgriffe des objektorientierten Design
erforderlich. Diese sollen hier nun kurz skizziert werden:

Eines der Grundprinzipien des objektorientieren Entwurfs wird wohl
durch das folgende Sprichwort treffend beschrieben:
\begin{quotation}
,,Encapsulate the concept that varies.``
\end{quotation}
Dementsprechend sollte man die Teile eines Programmes, welche of
ausgetauscht oder ver\"andert werden, durch klar definierte
Schnittstellen kapseln. Somit bietet es sich an, f\"ur die
Unterst\"utzung der neuen Dateiformate eine neue Klassenhierarchie zu
bilden, welche die Strategien zum Einlesen und zum Ausgeben der neuen
Formate enth\"alt. Je nach dem, welches Format zur Laufzeit
verarbeitet werden soll, lie{\ss}en sich Ein- und Ausgabeeinheit mit
einer konkreten Strategie konfigurieren, welche die formateigene Logik
in sich tr\"agt. Die Kommunikation mit der Strategie w\"urde dann
\"uber die feste Schnittstelle einer abstrakten Klasse ablaufen, damit
Vorg\"ange und Netzpl\"ane unabh\"angig von der tats\"achlich
eingesetzten Strategie
verarbeitet werden k\"onnen. Neue Formate lie{\ss}en sich auf diese
Weise leicht durch Unterklassenbildung einer weiteren konkreten
Strategie hinzuf\"ugen, ohne andere Teile des Programms \"andern zu
m\"ussen.\footnote{Die hier dargelegten Ausf\"uhrungen beziehen sich
  im Wesentlichen auf das ,,Strategy-Pattern``, welches in
  \textit{Design Patterns. Elements of Reusable Object-Oriented
    Software.} detailliert beschrieben wird.}

