\chapter[Abweichungen und Erg"anzungen zum Vorentwurf]{Abweichungen und Erg"anzungen\\zum Vorentwurf}
\label{Abweichungen}

Die \"Anderungen und Abweichungen, die sich zwischen dem Vorentwurf
und der aktuellen Version ergeben haben, seien hier anhand der Klassen
und Methoden aufgef\"uhrt, in denen sie aufgetreten sind.

\section{\"Anderungen in der Klasse Netzplan}

\subsection{Abbildung zwischen internen und externen Vorgangsnummern}

Ein neues Konzept, welches noch nicht im ersten Entwurf aufgef\"uhrt
wurde, bildet die in Abschnitt \ref{netzplan-logisch} beschriebene
Abbildung \(g\), welche die Vorgangsnummern auf die Folge der ersten
nat\"urlichen Zahlen abbildet. Innerhalb des Programms wurde dies
mithilfe der \texttt{Map<Integer, Integer> toInternal} realisiert,
welche zu jeder Vorgangsnummer den Index in der Liste aus Vorg\"angen
speichert (dies Bedeutet, dass der Wertebereich der nat\"urlichen
Zahlen in der Implementierung um die 0 erweitert wurde).
Dieses Vorgehen erm\"oglicht ein schnelles Zugreifen auf
Vorg\"anger und Nachfolger und sichert ebenfalls zu, dass beliebige
ganzzahlige Vorgangsnummern auftreten d\"urfen. Die Umkehrabbildung
\(g^{-1}\) ist durch \texttt{Map<Integer, Integer> fromInternal}
gegeben.

\subsection{Erweitertes Pr\"ufen der Nachfolger - Vorg\"anger -
  Relation}

W\"ahrend im Entwurf im Struktogramm zur Methode
\texttt{erzeugeAdjazenzen()} lediglich gepr\"uft wurde, ob f\"ur jeden
Vorgang die Beziehung zu seinen Nachfolgern konsistent ist, wird nun
auch sichergestellt, dass die Beziehung zu seinen Vorg\"angern
ebenfalls konsistent ist. Dieser Schritt ist -- insbesondere im
Hinblick auf die R\"uckw\"artsrechnung -- unerl\"asslich.

\subsection{Einsatz von Exceptions in den Methoden
  \texttt{istZyklenfrei()} und \texttt{istZusammenhaengend()}}

Um auch detaillierte Fehlermeldungen ausgeben zu k\"onnen, sind die im
Entwurf vorgeschlagenen Signaturen \texttt{private boolean istZyklenfrei()}
und \texttt{private boolean istZusammenhaengend()} nicht
ausreichend. Aus diesem Grund wird im Falle eines ung\"ultigen
Netzplanes bei diesen Methoden anstelle eines Wahrheitswertes eine
Exception geworfen.

\subsection{Queue aus Pfaden statt Liste aus Knoten in der Methode\\
  \texttt{istZyklenfrei()}}

Der im Entwurf vorgeschlagene Algorithmus zum Auffinden von Zyklen
musste leicht erweitert werden, damit er alle Zyklen korrekt erkennt
und auch Informationen \"uber diese preisgibt. Im beschriebenen
Algorithmus war es m\"oglich, dass die Zusammenf\"uhrung zweier Pfade
als Zyklus erkannt wird, da der Knoten, der beide Pfade vereinigt, von
beiden Vorg\"angern besucht wird. Dies wird verhindert, indem die
zugeh\"origen Pfade anstelle der aktuellen Knoten abgespeichert
werden. Weiterhin wurde in der Implementierung anstelle einer Liste
eine Queue gew\"ahlt, da sich diese besser eignet, wenn in jedem
Schritt das vorderste Element entfernt wird.

\subsection{Auslagerung der Rechenphasen in einzelne Methoden}

W\"ahrend es im Entwurf implizit vorgesehen war, die Rechenphasen
komplett im Konstruktor der Klasse Netzplan zu implementieren, wurden
diese nun in einzelne private Methoden ausgelagert, die den Quelltext
\"ubersichtlich halten.

\subsection{Verwendung von Queue statt Liste in den Rechenphasen}

Auch in den Rechenphasen wird nun statt einer Liste eine Queue
verwendet, um die aktuell zu bearbeitenden Knoten zu speichern. Diese
\"Anderung wurde wieder aus dem Argument heraus durchgef\"uhrt, dass
sich Queues besser eignen, wenn h\"aufig das erste Element entfernt
wird.

\section{\"Anderungen in der Klasse Vorgang}

\subsection{\"Uberschreiben der Methode \texttt{toString()}}

Damit einzelne Vorg\"ange leichter ausgegeben werden k\"onnen, wurde
in der Klasse Vorgang die Methode \texttt{toString()}
\"uberschrieben. So ist jeder Vorgang in der Lage, sich selbst als
String zu repr\"asentieren.
